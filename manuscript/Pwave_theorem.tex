\documentclass[aps,prd,twocolumn,tightenlines,letterpaper,nofootinbib]{revtex4-1}

\usepackage{amssymb,latexsym}
\usepackage{amsmath,amsbsy,bbm}
\usepackage{epsfig,bm,color}
\usepackage{graphicx,comment}
\usepackage[vcentermath]{youngtab}
\usepackage{slashed}
\usepackage{nicefrac}
\usepackage{tikz,pgfplots}

\unitlength=1mm

\usepackage{dutchcal}
\usepackage{calligra}

\DeclareMathAlphabet{\mathcalligra}{T1}{calligra}{m}{n}
\DeclareFontShape{T1}{calligra}{m}{n}{<->s*[2.2]callig15}{}
\newcommand{\scriptr}{\mathcalligra{r}\,}
\newcommand{\boldscriptr}{\pmb{\mathcalligra{r}}\,}

\begin{document}

\newcommand{\eftnopi}{\mbox{EFT$(\not \! \pi)$}}
\newcommand{\ve}[1]{\ensuremath{\boldsymbol{#1}}}
\newcommand{\ls}{\ve{L}\cdot\ve{S}}
\newcommand{\be}{\begin{equation}}
\newcommand{\ee}{\end{equation}}
\newcommand{\la}{\label}

\def\asy{\bm{\mathcal{A}}}
\def\ie{{\it i.e.~}}
\def\wrt{{\it w.r.t.~}}
\def\eg{{\it e.g.~}}
\def\he#1{{{}^#1\text{He}}}
\def\li#1{{{}^#1\text{Li}}}
\def\ov#1#2{\big\langle\,#1\,\big\vert\,#2\,\big\rangle}

\def\eqref#1{{(\ref{#1})}}

\def\yta{
\begin{tikzpicture}[scale=1.]
\draw[dashed,thick,opacity=0.5] (-2,-1.5) -- (7,-1.5);
\path
node at (-1,0)  [minimum size=1cm] {$\phi_0$}
node at (-1,-1) [minimum size=1cm] {$\phi_1$}
node at (-1,-3) [minimum size=1cm] {$\phi_\Omega$}
node at (0,0) [shape=rectangle,draw,minimum size=1cm] {1}
node at (1,0) [shape=rectangle,draw,minimum size=1cm] {2}
node at (2,0) [shape=rectangle,draw,minimum size=1cm] {3}
node at (3,0) [shape=rectangle,draw,minimum size=1cm] {4}
node at (4,0) [shape=rectangle,draw,minimum size=1cm] {5}
node at (5,0) [minimum size=1cm] {$\cdots$}
node at (6,0) [shape=rectangle,draw,minimum size=1cm] {$n_0$}
node at (0,-1) [shape=rectangle,draw,minimum size=1cm] {}
node at (1,-1) [shape=rectangle,draw,minimum size=1cm] {}
node at (2,-1) [minimum size=1cm] {$\cdots$}
node at (3,-1) [shape=rectangle,draw,minimum size=1cm] {}
node at (0,-2) [minimum size=1cm] {$\vdots$}
node at (2,-2) [rotate=90,minimum size=1cm] {$\ddots$}
node at (0,-3) [shape=rectangle,draw,minimum size=1cm] {}
node at (1,-3) [minimum size=1cm] {$\cdots$}
node at (2,-3) [shape=rectangle,draw,minimum size=1cm] {A}
;
\end{tikzpicture}
}

\newenvironment{ecce}[1][Ecce]{\begin{trivlist}\scriptsize\color{blue}
\item[\hskip \labelsep {\bfseries{\color{red} #1}}]}{\end{trivlist}}

\newenvironment{definition}[1][Definition]{\begin{trivlist}
\item[\hskip \labelsep {\textrm{\bfseries#1}}]}{\end{trivlist}}

 \author{L.~Contessi}
 \affiliation{Racah Institute of Physics, The Hebrew University, 91904, Jerusalem, Israel}
 \author{J.~Kirscher}
\affiliation{Theoretical Physics Group, Department of Physics and Astronomy, University of Manchester, Manchester, UK}

\title{No-bound-state theorem for $P$-wave systems} 

\begin{abstract}
We show that momentum-independent contact interactions can
stabilize composites of non-relativistic, identical fermions 
only in a totally symmetric, spatial state.
If the number of constituents exceeds the
number of accessible internal-symmetry states or another
reason precludes the total symmetry of the coordinate-space part of the
wave function, a zero-range $S$-wave interaction cannot bind
the system with respect to breakup into substructures.
This theorem holds for an arbitrary number of fermion species and spatial
dimensions. 
\end{abstract}
 
\maketitle

\paragraph{Overture}

The problem of a finite number of identical, point-like fermions
interacting with another is generic for many physical systems.
In the absence of other degrees of freedom, gauge fields for instance, the 
interaction between the fermions is purely due to
contact interactions. These contact interactions, however,
might exhibit an arbitrarily complicated momentum
dependence. The problem we address in the following is whether an 
interaction which comprises only those contact terms
without momentum dependence (zero-range $S$-wave interactions) can,
{\it in principle}, sustain a fermionic, bound $A$-body state with a
totally symmetric spatial wave function.

\paragraph{The no-bound-$P$-wave-state theorem}

We consider momentum-independent contact interactions pertinent for an
$A$-body system of the general structure

\be\la{eq.cont-int-gen}
\hat{V}=
\sum_{i\neq j\atop \widetilde{\ve{S}}}\hat{P}_2(\widetilde{\ve{S}})\delta^{(d)}_{ij}+
\sum_{i\neq j\neq k\atop \widetilde{\ve{S}}}\hat{P}_3(\widetilde{\ve{S}})
\delta^{(d)}_{ij}\delta^{(d)}_{jk}+
\ldots\;\;\;\;\;.
\ee

The spatial component is given by $d$-dimensional delta-functions,
$\delta^{(d)}_{ij}=\delta^{(d)}(\ve{r}_i-\ve{r}_j)$, which combine
with a dependence on the internal degrees of freedom (collectively
denoted by $\widetilde{\ve{S}}~\stackrel{~\text{\eg}}{=}(SU(2)~\text{spin, SU(N)~isospin})$).
For the operators corresponding to the latter,
$\hat{P}_{n\leq A}(\widetilde{\ve{S}})$,
we assume rank-$0$ tensors \wrt~to rotations in space,
and their commutation with the antisymmetrizing operator $\asy$
in order to have $\big[\hat{H},\asy\big]=0$.

For the wave function, we assume the existence of a complete orthonormal,
single-particle basis in which the state of the {\it interacting}
$A$-body system, with $(d)$ spatial and $(N)$ internal degrees of freedom
($\widetilde{r}\equiv(\ve{r},\ve{j})$), can be expanded:

\be\la{eq.wfkt-exp}
\Psi\left(\widetilde{\ve{r}}_1,\ldots,\widetilde{\ve{r}}_A\right)=
\sum_{\ve{n},\ve{m}}\left(c_{mn}\prod_{i=1}^A\phi_{n_i}(\ve{r}_i)~\xi_{m_i}
\right)\equiv\sum\Psi_{\ve{n}\ve{m}}\;\;\;,
\ee

with $\left\langle\Psi_{\ve{n}\ve{m}}\big\vert\Psi_{\ve{n}'\ve{m}'}
\right\rangle=\delta_{\ve{m}\ve{m}'}\delta_{\ve{n}\ve{n}'}$.
As we exclude totally symmetric states in space,
\mbox{$\exists~(i,j)~\vert~i\neq j~:~n_i\neq n_j$}. Thereby we exclude
distributions of all $A$ fermions into a single, spatial state $k$,
$\ve{n}=\delta_{ki}n_i$.

Next, we select one of the expansion terms, $\Psi_{\ve{n}\ve{m}}$,
arbitrarily. It is helpful, to 
visualize its behaviour \wrt~permutations with
a Young tableau. For $A=3$, \eg, we have
$\phi^{123}_{\ve{k}}\equiv\phi_a(1)\phi_a(2)\phi_b(3)=\young(12,3)$
in mind, with row labels $a$ and $b$ and two occupied internal states.
Particles $1$ and $2$ occupy the same spatial shell $a$, and hence
$\phi^{213(321)}_{\ve{k}}=(-)\phi^{123}_{\ve{k}}$. With an arbitrary
horizontal line,
we define a partition of the $A$ coordinates: an upper set $I$, and a lower
set $II$. The three-body example has only one such partition, namely,
$I=\lbrace 1,2\rbrace$ and $II=\lbrace 3\rbrace$. The Hamiltonian can always
be written as a sum of partition-internal operators, a kinetic-energy
term for the relative motion between the partition, and the interaction
between particles in different partitions:

\be\la{eq.cl-ham}
\hat{H}=\hat{H}_I+\hat{H}_{II}+\hat{T}_{\ve{R}}+\hat{V}_{I-II}\;\;\;.
\ee

We assume knowledge of the spectrum of the partitions
$\hat{H}_{I(II)}\Phi_{I(II)}=E_{I(II)}\Phi_{I(II)}$, which allows us to
write the stationary $A$-body Schr\"odinger equation for a single term
of the wave-function expansion

\be\la{eq.schroe-part}
\left(\hat{T}_{\ve{R}}+\hat{V}_{I-II}-E_\text{\tiny total}+E_I+E_{II}\right)
\asy\big(\Phi_I\Phi_{II}\phi_{\ve{R}}\big)=0\;\;\;.
\ee

It is crucial that conditions were neither made when selecting the term in
the expansion, \ie, the specific Young tableau, nor when partitioning
that term. A ``horizontal'' partition is always possible by the definition
of the system under investigation. Only those configuration with all 
particles in one shell cannot be partitioned. To those pure ``$S$-wave''
states, the theorem does not apply.

The projection onto $\Phi_I\Phi_{II}$ leads to the equation

\begin{align}\la{eq.schroe-pro}
\left\langle~\Phi_I\Phi_{II}~\big\vert\asy\left[\Phi_I\Phi_{II}\left(\hat{T}_{\ve{R}}-E_\text{\tiny total}+E_I+E_{II}
\right)\phi_{\ve{R}}\right]\right\rangle+\nonumber\\
\underbrace{\left\langle~\Phi_I\Phi_{II}~\big\vert~\hat{V}_{I-II}~\asy\left[\Phi_I\Phi_{II}\phi_{\ve{R}}\right]\right\rangle}_{\equiv V_\text{EX}}=0\;\;\;.%\nonumber
\end{align}

The subject of this article is to prove that

\be\la{eq.theorem}
V_\text{EX}=0\;\;\;{\text{for momentum-independent}\atop \text{zero-range interactions.}}
\ee

For an arbitrary tuple of interacting particles and a 
single-particle expansion as in \eqref{eq.wfkt-exp}, only
the identity $\bm{1}$ and certain permutations within the tuple
yield non-zero matrix elements. 
Any permutation which affects a particle which does not partake in
the interaction under consideration will be put into a state which
is orthogonal to the one it occupies in $\Phi_I\Phi_{II}$. As the
interaction cannot compensate this shift to another state, the
resulting matrix element is zero.

The contribution of the first term of \eqref{eq.cont-int-gen}~to
$V_\text{EX}$ is then
\begin{align}\la{eq.ex-2}
\ov{\Phi_I\Phi_{II}}{\hat{P}_2(\widetilde{\ve{S}})\delta^{(d)}_{ij}
~(\bm{1}-(i\leftrightarrow j))
\left[\Phi_I\Phi_{II}\phi_{\ve{R}}\right]}=\hspace{1cm}
\nonumber\\
\ov{\phi^{(i)}_{n_i}\phi^{(i)}_{n_j}}
{\phi^{(i)}_{n_i}\phi^{(i)}_{n_j}}\cdot\hspace{.7cm}
\nonumber\\
\Big[\ov{\xi^{(i)}_{m_i}\xi^{(j)}_{m_j}}{\hat{P}_2
~\big\vert~\xi^{(i)}_{m_i}\xi^{(j)}_{m_j}}-
\ov{\xi^{(i)}_{m_i}\xi^{(j)}_{m_j}}{\hat{P}_2
~\big\vert~\xi^{(j)}_{m_i}\xi^{(i)}_{m_j}}\Big]
\nonumber\\
\end{align}

where only the integration over the coordinates of particle $i$ does not
yield unity. The minus sign stems from the asymmetry of the spatial part
of the wave function, which follows from particles residing in different
partitions.
Hence, exchanging $(i\leftrightarrow j)$ in the spin part of the
wave function has no effect. Thus, last term of \eqref{eq.ex-2}~does
not change its value and thus cancels the other spin matrix element
for an arbitrary pair-interaction operator $\hat{P}_2$.

If we consider an arbitrary contribution to $V_\text{EX}$, \ie,
a term from \eqref{eq.cont-int-gen}~for a specific operator structure and
particle tuple,
it is not obvious that the matrix element
which involves not just one but an arbitrary number of pairs
vanishes, too. Instead of a single permutation, any permutation
($\hat{p}\in S_{|D|}$) of the symmetric group of the interacting set of particles
($D$), could yield non-zero matrix elements. The generalization of \eqref{eq.ex-2}~is

\begin{align}\la{eq.ex-2}
\sum_{p\in S_{|D|}}(-)^p\ov{\Phi_I\Phi_{II}}{\hat{P}_{|D|}
\prod_{i\neq j\in D}\delta^{(d)}_{ij}~\hat{p}~
\left[\Phi_I\Phi_{II}\phi_{\ve{R}}\right]}
\propto\hspace{1cm}
\nonumber\\
\sum_{p\in S_{|D|}}(-)^p\,\ov{\prod_{i\in D}\xi^{(i)}_{m_i}}{\hat{P}_{|D|}
~\big\vert~\prod_{i\in D}\xi^{\hat{p}(i)}_{m_{\hat{p}(i)}}}\;\;\;.
\nonumber\\
\end{align}

The $|D|!$ terms of this sum cancel as follows. 


\end{document}