\documentclass[aps,prd,twocolumn
,tightenlines,letterpaper,
%superscriptaddress,
nofootinbib]{revtex4-1}

\usepackage{amssymb,latexsym}
\usepackage{amsmath,amsbsy,bbm}
\usepackage{epsfig,bm,color}
\usepackage{graphicx,comment}
\usepackage[vcentermath]{youngtab}
\usepackage{slashed}
\usepackage{nicefrac}

\unitlength=1mm

\DeclareMathOperator{\st}{str}
\DeclareMathOperator{\tr}{tr}
\DeclareMathOperator{\Erfc}{Erfc}
\DeclareMathOperator{\Erf}{Erf}
\DeclareMathOperator{\Pf}{Pf}
\DeclareMathOperator{\sign}{sign}

\usepackage{dutchcal}
\usepackage{calligra}

\DeclareMathAlphabet{\mathcalligra}{T1}{calligra}{m}{n}
\DeclareFontShape{T1}{calligra}{m}{n}{<->s*[2.2]callig15}{}
\newcommand{\scriptr}{\mathcalligra{r}\,}
\newcommand{\boldscriptr}{\pmb{\mathcalligra{r}}\,}


\begin{document}

\newcommand{\eftnopi}{\mbox{EFT$(\not \! \pi)$}}
\newcommand{\ve}[1]{\ensuremath{\boldsymbol{#1}}}
\newcommand{\ecm}{\ensuremath{E_\text{\tiny cm}}}
\newcommand{\ls}{\ve{L}\cdot\ve{S}}
\newcommand{\ent}{~\widehat{=}~}

\def\a{{\alpha}}
\def\b{{\beta}}
\def\d{{\delta}}
\def\D{{\Delta}}
\def\X{{\Xi}}
\def\e{{\varepsilon}}
\def\g{{\gamma}}
\def\G{{\Gamma}}
\def\k{{\kappa}}
\def\l{{\lambda}}
\def\L{{\Lambda}}
\def\m{{\mu}}
\def\n{{\nu}}
\def\o{{\omega}}
\def\O{{\Omega}}
\def\S{{\Sigma}}
\def\s{{\sigma}}
\def\th{{\theta}}

\def\ol#1{{\overline{#1}}}


\def\Dslash{D\hskip-0.65em /}
\def\Dtslash{\tilde{D} \hskip-0.65em /}

\def\ie{{\it i.e.~}}
\def\wrt{{\it w.r.t.~}}
\def\eg{{\it e.g.~}}
\def\he#1{{{}^#1\text{He}}}
\def\li#1{{{}^#1\text{Li}}}
\def\QCPT{{Q$\chi$PT}}
\def\PQCPT{{PQ$\chi$PT}}
\def\tr{\text{tr}}
\def\str{\text{str}}
\def\diag{\text{diag}}
\def\order{{\mathcal O}}

\def\cF{{\mathcal F}}
\def\cG{{\mathcal G}}
\def\cE{{\mathcal E}}
\def\cS{{\mathcal S}}
\def\cC{{\mathcal C}}
\def\cB{{\mathcal B}}
\def\cT{{\mathcal T}}
\def\cQ{{\mathcal Q}}
\def\cL{{\mathcal L}}
\def\cO{{\mathcal O}}
\def\cA{{\mathcal A}}
\def\cQ{{\mathcal Q}}
\def\cR{{\mathcal R}}
\def\cH{{\mathcal H}}
\def\cW{{\mathcal W}}
\def\cM{{\mathcal M}}
\def\cD{{\mathcal D}}
\def\cZ{{\mathcal Z}}
\def\cN{{\mathcal N}}
\def\cP{{\mathcal P}}
\def\cK{{\mathcal K}}
\def\Qt{{\tilde{Q}}}
\def\Dt{{\tilde{D}}}
\def\St{{\tilde{\Sigma}}}
\def\cBt{{\tilde{\mathcal{B}}}}
\def\cDt{{\tilde{\mathcal{D}}}}
\def\cTt{{\tilde{\mathcal{T}}}}
\def\cMt{{\tilde{\mathcal{M}}}}
\def\At{{\tilde{A}}}
\def\cNt{{\tilde{\mathcal{N}}}}
\def\cOt{{\tilde{\mathcal{O}}}}
\def\cPt{{\tilde{\mathcal{P}}}}
\def\cI{{\mathcal{I}}}
\def\cJ{{\mathcal{J}}}

\def\eqref#1{{(\ref{#1})}}

\newenvironment{ecce}[1][Ecce]{\begin{trivlist}\scriptsize\color{blue}
\item[\hskip \labelsep {\bfseries{\color{red} #1}}]}{\end{trivlist}}

\newenvironment{definition}[1][Definition]{\begin{trivlist}
\item[\hskip \labelsep {\textrm{\bfseries#1}}]}{\end{trivlist}}

 \author{J.~Kirscher}
 \author{L.~Contessi}
%\email[]{$\texttt{kirschjs@web.de}$}
\affiliation{
$\mathcal{HUJI}$ (Humour University's Joke Institute)
}

 
\title{
A nucleon contact theory for P-wave dominated nuclei
} 

\begin{abstract}
We analyse \eftnopi~at leading order in few-nucleon systems which are
characterized by an amplitude pole whose imaginary part is small relative
\end{abstract}

\pacs{12.34.Ab}
 
\maketitle

\section{What is a P-wave state?}
First, the imaginary part of the pole location which characterises the state
must be small compared with the energy difference to the lowest threshold
at which the creation of particles which are not element of the theory
becomes energetically possible. Second, the state cannot reside in a totally
symmetric spatial configuration. \begin{ecce}I would like to include an explicit calculation
which relates an $>2$-body wave function, in some relative-coordinate basis, to a single-particle
basis which can be identified with a young tableaux. I think the technical term for that
is an inverse {\it Talmi transformation}.\end{ecce}

\begin{align*}
\yng(4) &\qquad \he{4} & \text{no P-wave state}\\
 &&{\exists}\lim_{\Lambda\to\infty}\eftnopi\\
\yng(4,1) &\qquad \he{5} & \text{P-wave state}\\
 &&\slashed{\exists}\lim_{\Lambda\to\infty}\eftnopi\\
\yng(4,2) &\qquad \li{6} & \text{P-wave state}\\
 &&\slashed{\exists}\vee{\exists}\lim_{\Lambda\to\infty}\eftnopi\\
 \yng(2,1) &\qquad {}^3n & \text{P-wave tri-neutron state}\\
 &&\slashed{\exists}\lim_{\Lambda\to\infty}\eftnopi^{\text{[we show]}}\\
  \yng(2,2) &\qquad {}^4n & \text{P-wave tetra-neutron state}\\
 &&\slashed{\exists}\lim_{\Lambda\to\infty}\eftnopi^{[\text{ref needed]}}\\
\end{align*}

For $\he{5}$, \eftnopi~cannot sustain anything but a free $\he{4}$-neutron
continuum state because any contribution from the fifth particle vanishes
in the zero-range limit. From the tableaux, that particle is in an odd partial
wave, and matrix elements of those are zero for contact interaction which do
not depend on the momentum. $\li{6}$, on the other hand, might be bound because
of the additional non-zero contribution from the symmetric pair.

In order for an unpaired nucleon to yield a pole structure which differs from
that which is given by the ``paired'' part, one has to augment the interaction
such that its zero-range limit is non-zero for the matrix element
$$\left\langle~\yng(1,1)~\left\vert\hat{H}_\text{LO}~\right\vert~\yng(1,1)~\right\rangle\;\;.$$

The additional term, $\hat{H}_\text{LO}=\hat{H}^{\slashed{\pi}}_\text{LO}+\delta\hat{H}$,
must not obstruct, through its iteration, S-wave observables which \eftnopi~is
devised to describe at leading order.

The following structure satisfies these constraints if its coupling constants
are renormalized properly (and I have, as of now, only a faint idea what that
means and how to achieve it) with a set of operators of identical, minimal mass
dimension:

\section{${}^2n$-$n$ and $\he{5}$ with \eftnopi~and $\eftnopi^*$}
First, one observes that the \eftnopi~induces an effective attraction between the neutron projectile
and the di-neutron and $\he{4}$ target, respectively, which is non-zero even if $\Lambda\to\infty$.
The pertinent phase shifts do clearly rule out the possibility of a resonance at low energy.

If we invoke $\eftnopi^*$ and calibrate its LEC's to ???, a steep rise in the phase shifts below
an energy at which the $\alpha$'s structure is resolved is stable in the limit $\Lambda\to\infty$.

\section{On the effect of an enhanced 2-body interaction in an A-body system}

Above, we showed results of the dependence of the dineutron-neutron and
4-helium-neutron spectra on the zero-range limit of a momentum-independent
2-body interaction. It remains to be shown that the found absence of any kind
of pole of the amplitude is robust \wrt to the renormalization of the three
interaction parameters. But, assuming that, \eg, a replacement of the deuteron
binding energy with the triplet-S-wave $np$ scattering length as input to
one of the LECs does not produce a pole in the P-wave structures considered
above, the interaction {\it must} be augmented in order to represent a useful
theory for the description of systems which fall into the $\he{5}$-class.

\begin{definition}
We assign all systems which
exhibit correlated behaviour which can be observed at scales which are too
small to resolve substructure of its constituting elements
{\bf and} can only be adequately described by a spatial 
configuration which is asymmetric under the exchange of all particles which
occupy {\it different} single-particle states (``shell shuffling'')
to the \underline{$\he{5}$-class}.
\end{definition}

As we are interested primarily in the description of $\he{5}$ itself, in-
and outgoing states contain P-waves. Thus, we assume that the same interaction
which is constrained in the 2-nucleon sector by P-wave data, namely \eftnopi~
at next$^2$-to-leading order -- due
to the large S-wave scales one set of momentum dependent operators is promoted
over the other, $\ve{q}^2$ is next-to-leading order while $\ve{q}\cdot\ve{q}'$
is next$^2$-to-leading order if understood as part of an amplitude -- is minimally
needed.

Is 2-body P-wave data correlated with $\he{5}$? We want to test this hypothesis
by including the N$^2$LO operator structure in the LO \eftnopi~potential.
Thereby, the two 2-body and the one 3-body contact interactions are iterated
along a tensor, a spin-orbit, and a $\ve{r}^2$ operator.

In a first empirical study, we were tempted to assess the dependence of the
$\alpha$-neutron system on the iteration of solely the spin-orbit term.
The pertinent phase shifts were found remarkably robust, and a significant change
was only observed at coupling strengths were the 2-body sector developed a
bound state in some P-wave channel (for \eftnopi-LO with $c_{LS}<0$, in the
${}^2P_0$ channel).
To understand this, consider the triton, a $J^\pi=\frac{1}{2}^+$ state, which
has small, but non-zero overlap with a state in which both relative motions
reside in P orbitals. Hence, the spin-orbit force affects the triton.
This effect, however, was found also of significance for much stronger
spin-orbit strengths, than those which bind the 2-body system. It is
counter intuitive that a system with more pairs in an attractive configuration
is more robust.

To be concrete, consider the spin-orbit force in the 2-body system:

$$c_{ls}\langle{}^3P_0\vert\ls\vert{}^3P_0\rangle=-2~c_{ls}\;\;.$$

Once this attraction becomes large enough to form a pocket in the repulsive
angular momentum well, the system develops, first a virtual or resonant, and
from it eventually a bound-state pole.
For another pole to emerge in the 3-nucleon amplitude, the pertinent matrix
element is

$$\left\langle\left[0^-\otimes \frac{1}{2}^+\right]^{\frac{1}{2}^+}\right\vert\ls\left\vert\left[0^-\otimes \frac{1}{2}^+\right]^{\frac{1}{2}^+}\right\rangle=-1~c_{ls}\;\;.$$

In words, when increasing the spin-orbit force, the formation of a bound P-wave
dimer does not stabilize the triton because the interaction is weaker between the
dimer and the third nucleon and, from this rough estimate, must be twice as
large to form a bound dimer between the $0^-$ neutron-proton and a neutron.
At this point, it is also understandable how increasing the strength of a 2-body
interaction does not necessarily stabilize any larger system. The interaction 
between a specific $(SL)J$-dimer and the other parts of the system, here the third
neutron might become repulsive. Here, this precludes the formation of a
$J^\pi=\frac{3}{2}^+$ $nnp$ state via this mechanism.

So how can the combination of spin-orbit, tensor, and $\ve{r}^2$ interaction
yield an effect at some strength in systems of different size?
If we would have singled out the tensor operator for the above exercise -- this
is complicated because of the constraint of the invariant deuteron state but
assuming that this is then taken into consideration with $\ve{r}^2$, we can
consider
$$c_{T}\langle{}^3P_0\vert Y_2\vert{}^3P_0\rangle=2~c_{T}\;\;,$$
while it is zero for this specific dimer-neutron matrix element.
The tensor can therefore be used to balance the effect of the spin-orbit term
in the 2-body sector, while it is ineffectual in the 3-body case.

\section{Search for the minimal nucleon-nucleon theory}
We demand that the theory shall be constrained with $A\leq 4$-body data, and thereby
is constructive in its prediction of a pole in the 5-body system which is within the
convergence radius of the newly construed EFT.
\paragraph{Operator structure}
We postulate  
\begin{align*}
\hat{V}=&
   \left[C_1^\Lambda~\hat{P}({}^1S_0) +            C_2^\Lambda~\hat{P}({}^3S_1)\right]~e^{-\frac{\Lambda^2}{4}r_{ij}^2}\\
&+              D_0^\Lambda~\hat{P}(S=\nicefrac{1}{2})~e^{-\frac{\Lambda^2}{4}\left(r_{ij}^2+r_{ik}^2\right)}\\
&+\left[C_3^\Lambda~\hat{P}({}^1S_0) +            C_4^\Lambda~\hat{P}({}^3S_1)\right]~\ve{r}^2~e^{-\frac{\Lambda^2}{4}r_{ij}^2}\\
&+-\frac{i}{2}C_5~(\ve{r}\times\ve{\nabla})\cdot(\ve{\sigma}_1+\ve{\sigma}_2)~e^{-\frac{\Lambda^2}{4}r_{ij}^2}\\
&+C_6~\left[\ve{\sigma}_1\cdot\ve{r}~\ve{\sigma}_2\cdot\ve{r}-\frac{1}{3}\ve{\sigma}_1\cdot\ve{\sigma}_2~\ve{r}^2\right]~e^{-\frac{\Lambda^2}{4}r_{ij}^2}
\end{align*}
as the minimal theory which predicts a P-wave state in the 5-body system, yields RG-invariant results
for $A\leq 4$ observables which are represented by amplitude poles whose location is close enough
to the respective threshold, that only particles which are either explicit DoF or resemble renormalization
conditions, c.f., the deuteron, the triton, or the virtual $nn$ singlet, can be produced.
\paragraph{Renormalization}
In addition to the three \eftnopi-LO LECs come 4 constants which correlate to 4 independent $nn$ P-wave
channels, and the phenomena which must be RG invariant are represented by the following list of observables:
\begin{description}
\item[structure-full channels]\mbox{}\vspace{5pt}\\ $\left\lbrace{}^1S_0,{}^3S_1,\epsilon\right\rbrace_{nn}$ and $\left\lbrace J^\pi=\frac{1}{2}^+\right\rbrace_{nnn}$
\item[structure-empty channels]\mbox{}\vspace{5pt}\\
$\left\lbrace{}^{1(3)}P_{1(0,1,2)}\right\rbrace_{nn}$ and $\left\lbrace J^\pi=\frac{1(3)}{2}^{-(\pm)}\right\rbrace_{nnn}$
\end{description}
Comments are in order. The interaction enables transition between S- and D-waves, and thus the
deuteron is already at LO a superposition of to partial waves. The number of constraints exceeds the
number of LEC's and hence the absence of a pole must be a consequence of the location of the included
poles, and the structure of the interaction. A trivial example is the vanishing $nn$ P-wave phase shift
at \eftnopi-LO. Before considering, \eg, why the augmented theory bears the chance to be ``well behaved''
in channels where there must be no shallow poles, we discuss the mechanism which protects the  $nnn$
$\nicefrac{3}{2}^+$ channel from sustaining a shallow pole in the zero-range limit of \eftnopi.

In the $nnp$ system, total spin $\nicefrac{3}{2}^+\ent\yng(3)_s$ which cannot be occupied in a totally
symmetric spatial state $\yng(3)_r$ because the corresponding iso-spin state would have to be anti-symmetric
in all three pairs, which is impossible for $T=\frac{1}{2}$.
So, our argument above applies here, too. In an arbitrary single-particle basis, the state can be expressed
in terms of vectors with the permutation-group structure
\begin{align}\label{eq.nnp-tripl}
    \left\vert~(nnp)\nicefrac{3}{2}^+~\right\rangle=&\yng(3)_s\otimes\yng(2,1)_r\otimes\yng(2,1)_I~+\nonumber\\
    &\yng(3)_s\otimes\yng(1,1,1)_r\otimes\yng(3)_I\;\;\;.
\end{align}
In the zero-range limit, matrix elements in this basis vanish for the interacting pair residing in an
asymmetric configuration. What remains is an exchange interaction.
Note that the positive parity does not preclude $\yng(1,1,1)_r$ or $\yng(2,1)_r$, because the single-particle
shells which relate to the rows of the tableaux might be of either parity.

To illustrate the potential failure of the augmented theory, we assume its LEC's to be calibrated
to all the structure-full observables and three $nn$ P-waves. The latter we realize by demanding that
a phase shift at some $\ecm<1~$MeV reproduces data. Thereby, a weak interaction is enforced.
Whereas the momentum-independent contact terms vanish, matrix elements for an interacting asymmetric
pair with the additional terms contribute


\section{conclusion}
Now we conjecture that this interaction produces non-trivial, \ie, not just projectile-target
continuum amplitudes for: \begin{ecce}list of observables we deem amenable to this theory, \eg, $\li{6}$ and ${}^{16}$O\end{ecce} 





\newpage
\bibliographystyle{unsrt}
\bibliography{Thebibliography.bib}

\end{document}