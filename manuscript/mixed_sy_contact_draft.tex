\documentclass[aps,prd,onecolumn
,tightenlines,letterpaper,
%superscriptaddress,linenumbers,
notitlepage,11pt,
nofootinbib]{revtex4-1}
\pagestyle{empty}
\usepackage{hyperref}
\usepackage{amssymb,latexsym}
\usepackage{amsmath,amsbsy,bbm}
\usepackage{epsfig,bm,color}
\usepackage{graphicx,comment}
\usepackage[vcentermath]{youngtab}
\usepackage{slashed}
\usepackage{nicefrac}
\usepackage{array}
\usepackage{cjhebrew}
\usepackage{amsthm}
\usepackage{bbold}
\usepackage{tikz,pgfplots}
\usepackage{etoolbox}
\usepackage{rotating}

\usetikzlibrary{positioning}
\usetikzlibrary{decorations.pathmorphing}
\usepgfplotslibrary{patchplots}
\usepgfplotslibrary{colormaps,fillbetween}

\newcommand*{\mprime}{^{\prime}\mkern-1.2mu}
\newcommand*{\mdprime}{^{\prime\prime}\mkern-1.2mu}
\newcommand*{\mtprime}{^{\prime\prime\prime}\mkern-1.2mu}

\makeatletter
\newcommand{\changeoperator}[1]{
  \csletcs{#1@saved}{#1@}
  \csdef{#1@}{\changed@operator{#1}}
}
\newcommand{\changed@operator}[1]{
  \mathop{
    \mathchoice{\textstyle\csuse{#1@saved}}
               {\csuse{#1@saved}}
               {\csuse{#1@saved}}
               {\csuse{#1@saved}}
  }
}
\makeatother

\changeoperator{sum}
\changeoperator{prod}

\definecolor{blue}{HTML}{4169E1}
\definecolor{red}{HTML}{DC143C}
\definecolor{green}{HTML}{2E8B57}
\definecolor{black}{HTML}{000000}
\definecolor{g1}{HTML}{A9A9A9}
\definecolor{g2}{HTML}{696969}
\definecolor{g3}{HTML}{7F7F7F}
\definecolor{g4}{HTML}{D3D3D3}

\DeclareMathOperator{\st}{str}
\DeclareMathOperator{\tr}{tr}
\DeclareMathOperator{\Erfc}{Erfc}
\DeclareMathOperator{\Erf}{Erf}
\DeclareMathOperator{\Pf}{Pf}
\DeclareMathOperator{\sign}{sign}

\usepackage{dutchcal}
\usepackage{calligra}

\DeclareMathAlphabet{\mathcalligra}{T1}{calligra}{m}{n}
\DeclareFontShape{T1}{calligra}{m}{n}{<->s*[2.2]callig15}{}
\newcommand{\scriptr}{\mathcalligra{r}\,}
\newcommand{\boldscriptr}{\pmb{\mathcalligra{r}}\,}


\begin{document}

\newcommand{\sleq}{{\tiny \sum\limits_i^{A+1}}}
\newcommand{\sube}{_\text{\tiny EX}}
\newcommand{\subd}{_\text{\tiny D}}
\newcommand{\la}{\label}
\newcommand{\eftnopi}{\mbox{EFT$(\not \! \pi)$}}
\newcommand{\ve}[1]{\ensuremath{\boldsymbol{#1}}}
\newcommand{\vcl}[1]{\ensuremath{\bar{\boldsymbol{r}}_\text{\tiny #1}}}
\newcommand{\cl}[1]{\ensuremath{\bar{r}_\text{\tiny #1}}}
\newcommand{\vsp}[1]{\ensuremath{\boldsymbol{r}}_\text{\tiny #1}}
\newcommand{\ls}{\ve{L}\cdot\ve{S}}
\newcommand{\bra}{\big\langle}
\newcommand{\ket}{\big\rangle}
\newcommand{\hl}{\big\vert}
\newcommand{\red}[1]{\textcolor{red}{#1}}
\newcommand{\green}[1]{\textcolor{green}{#1}}
\newcommand{\blue}[1]{\textcolor{blue}{#1}}
\newcommand{\drei}[1]{\delta^{(3)}\!\big(#1\big)}
\newcommand{\ddrei}[1]{\delta_{\tiny \Lambda}^{(3)}\!\big(#1\big)}

\def\a{\text{\cjRL{'}}}
\def\b{\text{\cjRL{b}}}
\def\d{\text{\cjRL{d}}}
\def\D{\text{\cjRL{g}}}
\def\X{\text{\cjRL{h}}}

\def\ol#1{{\overline{#1}}}

\def\nexists{\slashed{\exists}}
\def\Dslash{D\hskip-0.65em /}
\def\Dtslash{\tilde{D} \hskip-0.65em /}

\def\fm{\text{fm}^{-1}}
\def\ie{{\it i.e.~}}
\def\wrt{{\it w.r.t.~}}
\def\eg{{\it e.g.~}}
\def\he#1{{{}^#1\text{He}}}
\def\li#1{{{}^#1\text{Li}}}
\def\QCPT{{Q$\chi$PT}}
\def\PQCPT{{PQ$\chi$PT}}
\def\tr{\text{tr}}
\def\str{\text{str}}
\def\diag{\text{diag}}
\def\order{{\mathcal O}}

\def\cF{{\mathcal F}}
\def\cG{{\mathcal G}}
\def\cE{{\mathcal E}}
\def\cS{{\mathcal S}}
\def\cC{{\mathcal C}}
\def\cB{{\mathcal B}}
\def\cT{{\mathcal T}}
\def\cQ{{\mathcal Q}}
\def\cL{{\mathcal L}}
\def\cO{{\mathcal O}}
\def\cA{{\mathcal A}}
\def\cQ{{\mathcal Q}}
\def\cR{{\mathcal R}}
\def\cH{{\mathcal H}}
\def\cW{{\mathcal W}}
\def\cM{{\widehat{M}}}
\def\cD{{\mathcal D}}
\def\cZ{{\mathcal Z}}
\def\cN{{\mathcal N}}
\def\cP{{\mathcal P}}
\def\cK{{\mathcal K}}
\def\Qt{{\tilde{Q}}}
\def\Dt{{\tilde{D}}}
\def\St{{\tilde{\Sigma}}}
\def\cBt{{\tilde{\mathcal{B}}}}
\def\cDt{{\tilde{\mathcal{D}}}}
\def\cTt{{\tilde{\mathcal{T}}}}
\def\cMt{{\tilde{\mathcal{M}}}}
\def\At{{\tilde{A}}}
\def\cNt{{\tilde{\mathcal{N}}}}
\def\cOt{{\tilde{\mathcal{O}}}}
\def\cPt{{\tilde{\mathcal{P}}}}
\def\cI{{\mathcal{I}}}
\def\cJ{{\mathcal{J}}}


\def\eqref#1{{(\ref{#1})}}

\newcommand{\duo}[2]{\ensuremath{\arraycolsep=4pt\def\arraystretch{2}
	\begin{array}{c}
	#1\\
	#2
	\end{array}}
}
\renewcommand\thefootnote{\fnsymbol{footnote}}
\newenvironment{ecce}[1][Ecce]{\begin{trivlist}\scriptsize\color{blue}
\item[\hskip \labelsep {\bfseries{\color{red} #1}}]}{\end{trivlist}}

\newenvironment{definition}[1][Definition]{\begin{trivlist}
\item[\hskip \labelsep {\textrm{\bfseries#1}}]}{\end{trivlist}}

 \author{M.~Sch\"afer}
 \author{L.~Contessi}
 \author{J.~Kirscher}
%\email[]{$\texttt{kirschjs@web.de}$}
\affiliation{
$\mathcal{HUJI}$ (Humour University's Joke Institute)
}


\title{
On the character of contact interactions in few-body systems with mixed symmetry
}
\begin{abstract}
We analyse predictions of regularized two- and three-body contact interactions
in $A=2\ldots8$ body systems. The potentials are renormalized via 
spatially symmetric two- and three-body wave functions.
We apply these interactions to equal-mass systems with more constituents than
internal degrees of freedom. The spatial wave functions is consequently
of mixed symmetry.

We find regulator-induced characteristic interaction ranges for these systems
such that below a certain critical cutoff $\Lambda\lesssim\lambda_A$ a stable state exists.
For any $\Lambda\gtrsim\lambda_A$, the system is unstable
with respect to its lowest breakup threshold.
We find a linear relation between the critical  $\lambda_A\approx\b\cdot A$, \ie, a linear dependence of the critical
effective range on the number of constituents. This relation and the proximity of the interactions, which were used to obtain it, to unitarity,
support the first main conclusion of this work: Any momentum-independent, spherically symmetric interaction, which is
constrained by shallow two- and three-particle states, \ie, by totally spatially symmetric states, and a non-zero effective range


 Furthermore, we find no evidence for a resonance pole close to the scattering
threshold in the limit $\Lambda\to\infty$.

\end{abstract}

%\pacs{12.34.Ab}

\maketitle

\section{Overture}

Contact interactions stabilize the unitary $A$-boson system
with respect to certain thresholds which are set by an infinite
number of stable 3-body states, each of which is correlated with a
pair (one shallow and one deep) of 4-body states.
A pair of 5-body states is then found for each of the deep 4-body states.
This pattern generalizes such that a stable pair of a deep, and a shallow $A+1$-body state
is found for each $A$-body state which is deep relative to its $A-1$-body threshold.
With each particle, the binding energy increases with the number of pairs
in the system which interact via a unitary attraction. Nothing hinders all particles
to reside in the state which maximises the attraction.
These ``bosonic'' states are the ground states of fermions whose internal space
is of some dimension $n\geq A$\footnote{For nucleons $n=4$, \ie, two iso-spin and two spin degrees of freedom.}. The dynamics of fermionic systems with $n<A$ subject
to pairwise contact interaction is the subject of this note.

For
$\hat{V}=\sum\limits_{i<j}\lim\limits_{\lambda\to\infty}\delta_\lambda(\ve{r}_i-\ve{r}_j)$
we substantiate two conjecture with numerical evidence:

\newtheorem*{thm1}{Fermion contact instability}
\begin{thm1}
$\nexists$ stable $A$-body state ($A>N_\text{internal}$) for particles interacting
solely at zero-range, and there momentum independently (``contact interactions'').
\end{thm1}

For the theoretical description of nuclei, this theorem implies that the description of
6-helium, for example, must take into account the effective range of the nuclear interaction.

\newtheorem*{thm2}{Finite-range stability}
\begin{thm2}
$\forall$ non-zero, attractive, momentum-dependent (\ie, finite-range) interactions
$\exists A_c:\forall N\geq A_C,~\text{partitions}~F~\text{of}~N:\;B(N)>\sum\limits_{n\in F}B(n)$.
\end{thm2}
\newpage
\begin{table}
\setlength{\tabcolsep}{4pt}
\renewcommand{\arraystretch}{1.4}
\caption{\label{tab.legend}{Symbols, variables, numerical values as used in the manuscript..}}
\small\centering
\begin{tabular}{l|l}
\hline\hline
$\overline{\sum\limits_<}$ & $\sum\limits_{i=1}^{A-1}\vcl{i}$ \\
\hline
\end{tabular}
\end{table}


\begin{sidewaystable}
\setlength{\tabcolsep}{4pt}
\renewcommand{\arraystretch}{2.4}
\caption{\label{tab.rgmpot}{Defining parameters of the effective potential between
a Gaussian $A$-body core, characterized via the width $a$~\eqref{eq.corewfkt},
and one {\it odd} particle. The 2- and 3-body LECs $C^\Lambda_0$ and $D^\Lambda_1$ are
calibrated to a 2- and 3-body symmetric bound state (see table~\ref{tab.legend}). $A\mprime=A-1$}}
\small\centering
\begin{tabular}{lc|ccc}
\hline\hline
$i$ & $\eta_i$ & $\kappa_i$ & & \\
1   & $8~C_0^\Lambda~\frac{A\mprime}{\left(4+\frac{A\mprime}{A}\frac{\Lambda^2}{a}\right)^{3/2}}$  & $\frac{A\Lambda^2}{4A+A\mprime\frac{\Lambda^2}{a}}$ \\
2   & $
\frac{32 D_1^\Lambda a^3 A\mdprime A\mprime}{\left(16 a^2 A+4 a (3 A-1) \Lambda ^2+A\mdprime \Lambda ^4\right)^{3/2}}$ & $\frac{\Lambda ^2 \left(4 a^2 A+2 a A \Lambda ^2\right)}{16 a^2 A+4 a (3 A\mdprime+5) \Lambda ^2+A\mdprime \Lambda ^4}$ \\
3 & 
$\frac{32 D_1^\Lambda A\mdprime A\mprime}{\left(\frac{\left(4 a+\Lambda ^2\right) \left(4 a A+A\mdprime \Lambda ^2\right)}{a^2 A}\right)^{3/2}}$ & $\frac{2 a A \Lambda ^2}{4 a A+A\mdprime \Lambda ^2}$ \\
\hline
$n$ & $\zeta_i$ & $\alpha_n$ & $\beta_n$ & $\gamma_n$ \\
1 &$2 \sqrt{2} \left(\frac{A^2 \pi ^{A\mprime} a^{A\mdprime}}{A\mprime (A+1)^2}\right)^{3/2}$&
$\frac{a \left(A^3+A\right)}{2 A\mprime (A+1)^2}$&
$\frac{2 a A^2}{A\mprime (A+1)^2}$&
$\frac{a \left(A^3+A\right)}{2 A\mprime (A+1)^2}$\\
2 & 
$ \frac{8 C_0^\Lambda a^3 A\mprime A^{9/2}}{\pi ^{3/2} (A+1)^3 \left(4 a A\mprime+A\mdprime \Lambda ^2\right)^{3/2}}  $ & 
$\frac{\text{} a A \left(4 a \left(A^2+1\right)+\left(3 A^2+A+2\right) \Lambda ^2\right)}{2 (A+1)^2 \left(4 a A\mprime+A\mdprime \Lambda ^2\right)}$&
$\frac{4 \text{} a A^2 \left(2 a+\Lambda ^2\right)}{(A+1)^2 \left(4 a A\mprime+A\mdprime \Lambda ^2\right)}$&
$\frac{a A \left(4 a \left(A^2+1\right)+\left(A^2-A+2\right) \Lambda ^2\right)}{2 (A+1)^2 \left(4 a A\mprime+A\mdprime \Lambda ^2\right)}$ \\
3 &
$\frac{32 D_1^\Lambda A\mdprime A\mprime (a A)^{9/2}}{\pi ^{3/2} (A+1)^3 \left(16 a^2 A\mprime+4 a (3 A-4) \Lambda ^2+A\mtprime \Lambda ^4\right)^{3/2}}$&
$\frac{\text{} a A \left(16 a^2 \left(A^2+1\right)+4 a \left(5 A^2+A+4\right) \Lambda ^2+\left(5 A^2+2 A+3\right) \Lambda ^4\right)}{2 (A+1)^2 \left(16 a^2 A\mprime+4 a (3 A-4) \Lambda ^2+A\mtprime \Lambda ^4\right)}$&
$\frac{2 \text{} a A^2 \left(16 a^2+16 a \Lambda ^2+3 \Lambda ^4\right)}{(A+1)^2 \left(16 a^2 A\mprime+4 a (3 A-4) \Lambda ^2+A\mtprime \Lambda ^4\right)}$&
$\frac{a A \left(16 a^2 \left(A^2+1\right)+4 a \left(3 A^2-A+4\right) \Lambda ^2+\left(A^2-2 A+3\right) \Lambda ^4\right)}{2 (A+1)^2 \left(16 a^2 A\mprime+4 a (3 A-4) \Lambda ^2+A\mtprime \Lambda ^4\right)}$\\ 
4 &
$\frac{32 D_1^\Lambda A\mdprime A\mprime}{\pi ^{3/2} \left(\frac{(A+1)^2 \left(4 a+\Lambda ^2\right) \left(4 a A\mprime+A\mtprime \Lambda ^2\right)}{a^3 A^3}\right)^{3/2}}$&
$\frac{\text{} a A \left(4 a \left(A^2+1\right)+\left(5 A^2+2 A+3\right) \Lambda ^2\right)}{2 (A+1)^2 \left(4 a A\mprime+A\mtprime \Lambda ^2\right)}$&
$\frac{2 \text{} a A^2 \left(4 a+3 \Lambda ^2\right)}{(A+1)^2 \left(4 a A\mprime+A\mtprime \Lambda ^2\right)}$&
$\frac{a A \left(4 a \left(A^2+1\right)+\left(A^2-2 A+3\right) \Lambda ^2\right)}{2 (A+1)^2 \left(4 a A\mprime+A\mtprime \Lambda ^2\right)}$\\
\end{tabular}
\end{sidewaystable}

\newpage
\bibliographystyle{unsrt}
\bibliography{Thebibliography.bib}

\end{document}
