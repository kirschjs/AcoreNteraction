\documentclass[aps,prd,onecolumn
,tightenlines,letterpaper,
%superscriptaddress,linenumbers,
notitlepage,11pt,
nofootinbib]{revtex4-1}
\pagestyle{empty}
\usepackage{hyperref}
\usepackage{amssymb,latexsym}
\usepackage{amsmath,amsbsy,bbm}
\usepackage{epsfig,bm,color}
\usepackage{graphicx,comment}
\usepackage[vcentermath]{youngtab}
\usepackage{slashed}
\usepackage{nicefrac}
\usepackage{array}
\usepackage{cjhebrew}
\usepackage{amsthm}
\usepackage{bbold}
\usepackage{tikz,pgfplots}
\usepackage{etoolbox}

\usetikzlibrary{positioning}
\usetikzlibrary{decorations.pathmorphing}
\usepgfplotslibrary{patchplots}
\usepgfplotslibrary{colormaps,fillbetween}

\makeatletter
\newcommand{\changeoperator}[1]{
  \csletcs{#1@saved}{#1@}
  \csdef{#1@}{\changed@operator{#1}}
}
\newcommand{\changed@operator}[1]{
  \mathop{
    \mathchoice{\textstyle\csuse{#1@saved}}
               {\csuse{#1@saved}}
               {\csuse{#1@saved}}
               {\csuse{#1@saved}}
  }
}
\makeatother

\changeoperator{sum}
\changeoperator{prod}

\definecolor{blue}{HTML}{4169E1}
\definecolor{red}{HTML}{DC143C}
\definecolor{green}{HTML}{2E8B57}
\definecolor{black}{HTML}{000000}
\definecolor{g1}{HTML}{A9A9A9}
\definecolor{g2}{HTML}{696969}
\definecolor{g3}{HTML}{7F7F7F}
\definecolor{g4}{HTML}{D3D3D3}

\DeclareMathOperator{\st}{str}
\DeclareMathOperator{\tr}{tr}
\DeclareMathOperator{\Erfc}{Erfc}
\DeclareMathOperator{\Erf}{Erf}
\DeclareMathOperator{\Pf}{Pf}
\DeclareMathOperator{\sign}{sign}

\usepackage{dutchcal}
\usepackage{calligra}

\DeclareMathAlphabet{\mathcalligra}{T1}{calligra}{m}{n}
\DeclareFontShape{T1}{calligra}{m}{n}{<->s*[2.2]callig15}{}
\newcommand{\scriptr}{\mathcalligra{r}\,}
\newcommand{\boldscriptr}{\pmb{\mathcalligra{r}}\,}


\begin{document}

\newcommand{\sleq}{{\tiny \sum\limits_i^{A+1}}}
\newcommand{\sube}{_\text{\tiny EX}}
\newcommand{\subd}{_\text{\tiny D}}
\newcommand{\la}{\label}
\newcommand{\eftnopi}{\mbox{EFT$(\not \! \pi)$}}
\newcommand{\ve}[1]{\ensuremath{\boldsymbol{#1}}}
\newcommand{\vcl}[1]{\ensuremath{\bar{\boldsymbol{r}}_\text{\tiny #1}}}
\newcommand{\cl}[1]{\ensuremath{\bar{r}_\text{\tiny #1}}}
\newcommand{\vsp}[1]{\ensuremath{\boldsymbol{r}}_\text{\tiny #1}}
\newcommand{\ls}{\ve{L}\cdot\ve{S}}
\newcommand{\bra}{\big\langle}
\newcommand{\ket}{\big\rangle}
\newcommand{\hl}{\big\vert}
\newcommand{\red}[1]{\textcolor{red}{#1}}
\newcommand{\green}[1]{\textcolor{green}{#1}}
\newcommand{\blue}[1]{\textcolor{blue}{#1}}
\newcommand{\drei}[1]{\delta^{(3)}\!\big(#1\big)}
\newcommand{\ddrei}[1]{\delta_{\tiny \Lambda}^{(3)}\!\big(#1\big)}

\def\a{\text{\cjRL{'}}}
\def\b{\text{\cjRL{b}}}
\def\d{\text{\cjRL{d}}}
\def\D{\text{\cjRL{g}}}
\def\X{\text{\cjRL{h}}}

\def\ol#1{{\overline{#1}}}

\def\nexists{\slashed{\exists}}
\def\Dslash{D\hskip-0.65em /}
\def\Dtslash{\tilde{D} \hskip-0.65em /}

\def\fm{\text{fm}^{-1}}
\def\ie{{\it i.e.~}}
\def\wrt{{\it w.r.t.~}}
\def\eg{{\it e.g.~}}
\def\he#1{{{}^#1\text{He}}}
\def\li#1{{{}^#1\text{Li}}}
\def\QCPT{{Q$\chi$PT}}
\def\PQCPT{{PQ$\chi$PT}}
\def\tr{\text{tr}}
\def\str{\text{str}}
\def\diag{\text{diag}}
\def\order{{\mathcal O}}

\def\cF{{\mathcal F}}
\def\cG{{\mathcal G}}
\def\cE{{\mathcal E}}
\def\cS{{\mathcal S}}
\def\cC{{\mathcal C}}
\def\cB{{\mathcal B}}
\def\cT{{\mathcal T}}
\def\cQ{{\mathcal Q}}
\def\cL{{\mathcal L}}
\def\cO{{\mathcal O}}
\def\cA{{\mathcal A}}
\def\cQ{{\mathcal Q}}
\def\cR{{\mathcal R}}
\def\cH{{\mathcal H}}
\def\cW{{\mathcal W}}
\def\cM{{\widehat{M}}}
\def\cD{{\mathcal D}}
\def\cZ{{\mathcal Z}}
\def\cN{{\mathcal N}}
\def\cP{{\mathcal P}}
\def\cK{{\mathcal K}}
\def\Qt{{\tilde{Q}}}
\def\Dt{{\tilde{D}}}
\def\St{{\tilde{\Sigma}}}
\def\cBt{{\tilde{\mathcal{B}}}}
\def\cDt{{\tilde{\mathcal{D}}}}
\def\cTt{{\tilde{\mathcal{T}}}}
\def\cMt{{\tilde{\mathcal{M}}}}
\def\At{{\tilde{A}}}
\def\cNt{{\tilde{\mathcal{N}}}}
\def\cOt{{\tilde{\mathcal{O}}}}
\def\cPt{{\tilde{\mathcal{P}}}}
\def\cI{{\mathcal{I}}}
\def\cJ{{\mathcal{J}}}


\def\eqref#1{{(\ref{#1})}}

\newcommand{\duo}[2]{\ensuremath{\arraycolsep=4pt\def\arraystretch{2}
	\begin{array}{c}
	#1\\
	#2
	\end{array}}
}
\renewcommand\thefootnote{\fnsymbol{footnote}}
\newenvironment{ecce}[1][Ecce]{\begin{trivlist}\scriptsize\color{blue}
\item[\hskip \labelsep {\bfseries{\color{red} #1}}]}{\end{trivlist}}

\newenvironment{definition}[1][Definition]{\begin{trivlist}
\item[\hskip \labelsep {\textrm{\bfseries#1}}]}{\end{trivlist}}

 \author{M.~Sch\"afer}
 \author{L.~Contessi}
 \author{J.~Kirscher}
%\email[]{$\texttt{kirschjs@web.de}$}
\affiliation{
$\mathcal{HUJI}$ (Humour University's Joke Institute)
}


\title{
Contact interactions and $P$-wave dominated few-body systems
}
\begin{abstract}
We analyse predictions of regularized two- and three-body contact interactions
in $A=2\ldots8$ body systems. The interactions are renormalized with observables
which are associated with totally symmetric spatial wave functions, specifically, the
deuteron and triton ground states.
We apply these interactions to systems whose number of fermionic constituents exceeds the number of
internal degrees of freedom, thereby precluding a totally symmetric spatial wave function.

For such $A$-body systems, we find a characteristic interaction range in form of a regulator parameter $\lambda_A\in\lbrace 0.01~\fm, \ldots,10~\fm\rbrace$
such that for $\Lambda\lesssim\lambda_A$ a stable state exists, while for any $\Lambda\gtrsim\lambda_A$ the system is unstable
with respect to its lowest breakup threshold. Specifically, we $\lambda_A\approx\b\cdot A$, \ie, a linear dependence of the critical
effective range on the number of constituents. This relation and the proximity of the interactions, which were used to obtain it, to unitarity,
support the first main conclusion of this work: Any momentum-independent, spherically symmetric interaction, which is
constrained by shallow two- and three-particle states, \ie, by totally spatially symmetric states, and a non-zero effective range


 Furthermore, we find no evidence for a resonance pole close to the scattering
threshold in the limit $\Lambda\to\infty$.

\end{abstract}

%\pacs{12.34.Ab}

\maketitle

\section{Overture}

Contact interactions stabilize the unitary $A$-boson system
with respect to certain thresholds which are set by an infinite
number of stable 3-body states, each of which is correlated with a
pair (one shallow and one deep) of 4-body states.
A pair of 5-body states is then found for each of the deep 4-body states.
This pattern generalizes such that a stable pair of a deep, and a shallow $A+1$-body state
is found for each $A$-body state which is deep relative to its $A-1$-body threshold.
With each particle, the binding energy increases with the number of pairs
in the system which interact via a unitary attraction. Nothing hinders all particles
to reside in the state which maximises the attraction.
These ``bosonic'' states are the ground states of fermions whose internal space
is of some dimension $n\geq A$\footnote{For nucleons $n=4$, \ie, two iso-spin and two spin degrees of freedom.}. The dynamics of fermionic systems with $n<A$ subject
to pairwise contact interaction is the subject of this note.

For
$\hat{V}=\sum\limits_{i<j}\lim\limits_{\lambda\to\infty}\delta_\lambda(\ve{r}_i-\ve{r}_j)$
we substantiate two conjecture with numerical evidence:

\newtheorem*{thm1}{Fermion contact instability}
\begin{thm1}
$\nexists$ stable $A$-body state ($A>N_\text{internal}$) for particles interacting
solely at zero-range, and there momentum independently (``contact interactions'').
\end{thm1}

For the theoretical description of nuclei, this theorem implies that the description of
6-helium, for example, must take into account the effective range of the nuclear interaction.

\newtheorem*{thm2}{Finite-range stability}
\begin{thm2}
$\forall$ non-zero, attractive, momentum-dependent (\ie, finite-range) interactions
$\exists A_c:\forall N\geq A_C,~\text{partitions}~F~\text{of}~N:\;B(N)>\sum\limits_{n\in F}B(n)$.
\end{thm2}

\begin{figure}
\[\arraycolsep=4pt\def\arraystretch{4}
\begin{array}{l|lllll}
AB\ldots Z &  (AB\ldots Z)A & (AB\ldots Z_\text{even})^n & (AB\ldots Z_\text{odd})^n & & \\
\duo{\yng(2) }{ {}^2n, np} &
\duo{\yng(2,1)}{ {}^3\text{n}, \left({}^3\text{He}\right)^3} &
\duo{\yng(2,2)}{ {}^4n} &
\duo{\yng(1,1)}{{}^2n(S=1)} &
\duo{\yng(3,2)}{{}^5\text{H(e)}} &
\duo{\yng(4,2)}{{}^6\text{Li}} \\
\duo{\yng(3)}{ {}^3\text{H(e)}} &
\duo{\yng(3,1)}{{}^4\text{H}} &
\duo{\yng(4,4)}{{}^8\text{Be}} &
\duo{\yng(3,3)}{{}^6\text{Li}} &
\yng(4,3) &
\yng(6,2) \\
\duo{\yng(4)}{{}^4\text{He}} &
\duo{\yng(4,1)}{{}^5\text{H(e)}} &
\duo{\yng(4,4,4)}{{}^{12}\text{C}} &
\yng(3,3,3) &  &  \\ \hline
\yng(5) &
\yng(5,1) & &  &  &  \\
\yng(6) &
\yng(6,1) & &  &  &  \\
\yng(7) & \yng(7,1) & &  &  &  \\
\yng(8) & \yng(8,1) & &  &  &  \\\hline
\begin{minipage}[c]{.15\textwidth}
generalization of the universal ratio $\lim\limits_{n\to\infty}\frac{B_n}{B_{n-1}}$ to
$\frac{B_n(A)}{B_{n}(A+1)}\stackrel{?}{=}f(A)$
\end{minipage} &
\begin{minipage}[c]{.15\textwidth} odd-odd (imbalanced)\\effect of additional bosons to which the $2^\text{\tiny nd}$ fermion can bind; no additional exchange effect from one system size to the other.
\end{minipage} &
\begin{minipage}[c]{.15\textwidth} even-even (balanced)\\exchange effects in combination
with increasing number of cross-fragmentinteractions.
\end{minipage} &
\begin{minipage}[c]{.15\textwidth}odd-odd (balanced) same as even-even (triplet vs. doublet)
\end{minipage} &
\begin{minipage}[c]{.15\textwidth}
\end{minipage} &
\begin{minipage}[c]{.15\textwidth}
\end{minipage} \\
\end{array}
\]\caption{\small Classification of $A$-body systems according to particle number and
accessible internal states.}
\end{figure}

\section{$\lambda_c$~speculation}

\begin{figure}

\begin{tikzpicture}
    \begin{axis}
    [
    scale = 1.95,
    %ymax=2.2,
    scaled ticks=false,
     %thick,
     smooth,
     %no markers,
     xlabel={$A$},
     xlabel style={yshift=-5ex, font={\fontsize{18 pt}{12 pt}\selectfont}},
     ylabel={$\lambda_c~$[fm$^{-1}$]},
     ylabel style={yshift=5ex, font={\fontsize{18 pt}{12 pt}\selectfont}},
     %legend entries={,,{ ${}^3P_0$},,,{ ${}^3P_1$},,,{ ${}^3P_2$}},
     legend columns=3,
     legend style={/tikz/column 4/.style={column sep=14pt,},/tikz/column 2/.style={column sep=14pt,},draw=none,row sep = 5.5pt,at={(.5,-.32)},anchor=center},
     xtick={2,3,4,5,6},
     %xticklabels={\small\textcolor{red}{A-A},\small\textcolor{red}{A-A}B,\small\textcolor{red}{A-A}BC,\small\textcolor{red}{A-A}BCD,\small\textcolor{red}{A-A}BCDE},
     xticklabel style={rotate=0,yshift=-1.5ex, font={\fontsize{18 pt}{12 pt}\selectfont}},
     yticklabel style={rotate=0,xshift=-1.ex, font={\fontsize{16 pt}{12 pt}\selectfont}},
     ]

\addplot[red!70!black,mark=*] table[x expr=\thisrowno{0},y expr={ \thisrowno{3}>=0 ? \thisrowno{3} : 1./0.0 }]
{lambda_crit.dat};
\addlegendentry{$\young(A,\ldots,A)~\hat{=}~$fermion$^{\textcolor{red}{n}}$}

\addplot[green!60!black,mark=*] table[x expr=\thisrowno{0},y expr={ \thisrowno{2}>=0 ? \thisrowno{2} : 1./0.0 }]
{lambda_crit.dat};
\addlegendentry{$\young(AB,A,\ldots,A)~\hat{=}~$fermion-(fermion,boson$^{\textcolor{red}{n}}$)}

\addplot[blue!70!black,mark=*] table[x expr=\thisrowno{0},y expr={ \thisrowno{1}>=0 ? \thisrowno{1} : 1./0.0 }]
{lambda_crit.dat};
\addlegendentry{$\young(AB\ldots Z,A)~\hat{=}~$fermion$^{\textcolor{red}{n}}$-(fermion,boson)}

    \end{axis}

\end{tikzpicture}
\caption{\small .}

\end{figure}

If two systems $A$ and $B$ have similar critical binding ranges, $\lambda_c(A)\approx\lambda_c(B)$ we indicate this with
$A\sim B$.

\newtheorem{obs}{Observation}

\begin{obs}[${\tiny\yng(1,1)}<{\tiny\yng(1,1,1)}$]
Although, the third particle is forced into a higher orbital excitation, it is stable for even shorter ranges compared with the two-fermion system.
The higher orbital suggests a higher angular-momentum barrier which demands an even longer-ranged attraction. Expressing the wave function in cluster
coordinates, which are in this case identical to Jacobi coordinates, the 
\end{obs}

\section{Resonating-group derivation of the core-particle interaction}

\begin{gather}
\bra\phi_A\hl\big(\hat{T}_{\ve{R}}+\hat{V}_\text{\tiny A,A+1}-E\big)\hat{A}\left[\phi_A\psi(R)\right]\ket=0\\
\Rightarrow~\left(\hat{T}_{\ve{R}}-E\right)\bra\phi_A\hl\hat{A}\left[\phi_A\psi\right]
\ket+\bra\phi_A\hl\hat{V}_\text{\tiny A,A+1}\hl\hat{A}\left[\phi_A\psi\right]\ket=0
\la{eq.rgm2}
\end{gather}
with an average over the internal coordinates of the fragment:
\begin{gather}
\bra\ldots\ket=\prod\limits_{i=1}^{A-1}\int d^3\vcl{i}
\intertext{which we chose to express relative to its center or mass, \ie, we use
the following set of coordinates}
\vcl{i}:=\vsp{i}-\frac{\scriptstyle \sum_{i=1}^A\vsp{i}}{A}\;\;\;i\in\lbrace 1,\ldots,A\rbrace\\
\ve{R}:=\vsp{A+1}-\frac{\scriptstyle \sum_{i=1}^A\vsp{i}}{A}\\
\ve{R}_\text{\tiny cm}:=\frac{\scriptstyle \sum_{i=1}^{A+1}\vsp{i}}{A+1}
\end{gather}
which comprises $A$ independent vectors because $\vcl{A}=-(\vcl{1}+\ldots+\vcl{A-1})$ for
$A+1$ equal-mass particles located at single-particle coordinates $\vsp{1,$\ldots$,A+1}$.
\begin{gather}
\phi_A:=e^{-\frac{a}{2}\sum_{i=1}^{\red{A}}\vcl{i}^2}=e^{-a\sum^{\red{A-1}}\vcl{i}^2-a\sum^{\red{A-1}}_{\red{i<j}}\vcl{i}\cdot\vcl{j}}\la{eq.wfktphi}\\
\psi\to\int d^3\ve{R}'~\drei{\ve{R}-\ve{R}'}\psi(\ve{R}')\la{eq.wfktpsi}\\
\hat{A}=\mathbb{1}-\hat{P}(\vsp{A}\leftrightarrow\vsp{A+1})
\end{gather}
\begin{align}
\hat{P}(\vsp{A}\leftrightarrow\vsp{A+1})\left[\ve{R}=\vsp{A+1}-\frac{\sum^A_i\vsp{i}}{A}\right]=&~-A^{-1}\ve{R}-(1+A^{-1})\sum_i^{A-1}\vcl{i}\\
\hat{P}(\vsp{A}\leftrightarrow\vsp{A+1})\left[\drei{\ve{R}-\ve{R}'}\right]=&~(2\pi)^{-3}\int ds~e^{is\left(A^{-1}\ve{R}+(1+A^{-1})\sum_i^{A-1}\vcl{i}+\ve{R}'\right)}\\
%
\hat{P}(\vsp{A}\leftrightarrow\vsp{A+1})\left[\phi_A\psi\right]=&~\frac{A}{A+1}
(2\pi)^{-3}\prod\limits_{m\in\lbrace x,y,z\rbrace}~\int\limits_{\mathbb{R}^2}
d(s,\ve{R}^{'}_m)~e^{-\frac{a\epsilon_1}{2}\ve{R}_m^2-is\left(\frac{\ve{R}_m}{A+1}+\frac{A}{A+1}\ve{R}^{'}_m\right)}\\
&~\cdot e^{-\frac{a\epsilon_1}{2}\sum\limits_{i<A}\vcl{m,i}^2-\frac{a}{A}\sum\limits_{i\neq j<A}\vcl{m,i}\vcl{m,j}+\frac{a\ve{R}_m}{A}\sum\limits_{i<A}\vcl{m,i}-is\sum\limits_{i<A}\vcl{m,i}}\blue{\times\psi(\ve{R}')}
\end{align}
\begin{align}
\hat{V}_\text{\tiny A,A+1}=&~C_0(\Lambda)\sum\limits_{\red{i\leq A}}\ddrei{\vsp{i}-\vsp{A+1}}\la{eq.v2}\\
&~+D_1(\Lambda)\sum\limits_{\red{i\neq j\atop \leq A}}\left\lbrace\ddrei{\vsp{i}-\vsp{A+1}}\ddrei{\vsp{j}-\vsp{A+1}}+\ddrei{\vsp{i}-\vsp{A+1}}\ddrei{\vsp{j}-\vsp{j}}\right\rbrace\la{eq.v3}
\end{align}
An effective potential which considers the
interaction potential -- which weighs a configuration of particles within a limited
parameter space as circumscribed by the effective theory -- and the
identity of particles -- reflected in the implied (anti)symmetry of the
wave function -- is defined by
\begin{gather}
\int\limits_{\mathbb{R}}d^3\ve{R'}~\hat{V}_\text{\tiny eff}(\ve{R},\ve{R}')\psi(\ve{R}')
:=
\bra\phi_A\hl\hat{V}_\text{\tiny A,A+1}\hl\hat{A}\left[\phi_A\psi\right]\ket-\left(\hat{T}_{\ve{R}}-E\right)\bra\phi_A\hl\hat{P}(\vsp{A}\leftrightarrow\vsp{A+1})\left[\phi_A\psi\right]\ket\;\;,
\end{gather}
a function of
$$ \big\lbrace\ve{R},\ve{R}',a,\Lambda,C_0,D_1,A\big\rbrace\;\;. $$
Unorthodoxly, we do thus not distinguish between overlap/exchange kernel and direct/exchange interaction.

\subsection{Overlap Kernel}
The so-called exchange kernel measures the overlap of the fragment wave function, on the left-hand side, and the total
wave function on the right. 
\begin{gather*}
\mathcal{K}(R):=\bra\phi_A\hl\hat{A}\left[\phi_A\psi\right]\ket\;\;.
\intertext{The demand for a totally antisymmetric wave function preserves the dependence of this quantity
on the quantum numbers which characterize the fragment state $\phi_A$. This dependence dissappears only in the
approximation $\hat{A}=\mathbb{1}$ for orthonormal $\phi_A$'s. Assuming that the particles with label $A$ and $A+1$,
respectively, are identical,}
\mathcal{K}(R)=\int d(\vcl{1\ldots A},\ve{R}')e^{-\frac{a}{2}\sleq\vcl{i}^2}\left(\mathbb{1}-\hat{P}(\vsp{A},\vsp{A+1})\right)\left[
e^{-\frac{a}{2}\sleq\vcl{i}^2}\drei{\ve{R}-\ve{R}'}\right]\psi(\ve{R}')\;\;.
\intertext{We find}
\hat{P}(\vsp{A},\vsp{A+1})\phi_A=\\
\hat{P}(\vsp{A},\vsp{A+1})\drei{\ve{R}-\ve{R}'}=\frac{A}{A+1}~\drei{\vcl{A}-\frac{\ve{R}}{A+1}-\frac{A}{A+1}\ve{R}'}\\
=(2\pi)^{-3}\prod\limits_{m\in\lbrace x,y,z\rbrace}\int_\mathbb{R}ds~e^{-is\left(\vcl{m,A}-\frac{\ve{R}_m}{A+1}-\frac{A}{A+1}\ve{R}^{'}_m\right)}
\end{gather*}
We proceed with
\begin{align}
\bra\phi_A\hl\hat{A}\left[\phi_A\psi\right]\ket=&
~\int d^3\ve{R}'~\sum\limits_{k\in\lbrace \text{D,EX}
\rbrace}A_k\left(
\int\limits_{\mathbb{R}}\ldots\int\limits_{\mathbb{R}}d(\cl{1,$\ldots$,A-1},s)~
e^{-\frac{1}{2}\vcl{}^\intercal\cM_k\vcl{}+\ve{\cS}^\intercal_k\cdot\vcl{}-isB_k}
\right)^{\vec{3} }~\blue{\times\psi(\ve{R}')}
\la{eq.exkernel}\\
=&~\left\lbrace
\drei{\ve{R}-\ve{R}'}
-\frac{A}{A+1}\left(\pi^{-1}(A-1)^{-1}\right)^{3/2}e^{-\alpha_k\ve{R}^2-\beta_k\ve{R}'^2-\gamma_k\ve{R}\cdot\ve{R}'}
\right\rbrace
\\
&~\times\left(\left(\frac{\pi}{a}\right)^{A-1}A^{-1}\right)^{3/2}~\psi(\ve{R}')\;\;,
\end{align}
with $\vec{3} $ indicating a product of prefactors$\times$exponentials
for the spatial coordinates, \ie, in the exponents, components are to be replaced by
vectors, and prefactors are $n$-cubed.
Note that the combinatoric factors $A_k$ are included only once and not for each 
dimension, separately.
\begin{gather}
A_\text{D(irect)}=1\;\;;\;\;\cM\subd =\texttt{MAT}(\text{\small A-1})=\begin{pmatrix}4a & & & \\ & 4a & &\multicolumn{2}{l}{\!\!\!\!(2a)_\nabla}\\
 \multicolumn{2}{c}{(2a)_\triangle} & \ddots & \\ & & & & 4a \\\end{pmatrix}
\;\;;\;\;\ve{\cS}\subd =\ve{0}\;\;;\;\;B\subd =(R-R')
\\
\det\cM\subd =2^{A-1}A~a^{A-1} \\
A_\text{EX(change)}=-\frac{A}{A+1}(2\pi)^{-3}\;\;;\;\;B\sube =
\left(\frac{R}{A+1}+\frac{A}{A+1}R^{'}\right)
\intertext{The $A$ dependence in $A\sube$ stems from a rescaling of $\delta(R,R')$,
and $(2\pi)^{-3}$ from the Fourier representation of it.}
\cM\sube =\begin{pmatrix}a\epsilon_3 & & & \\ & a\epsilon_3 & &\multicolumn{2}{l}{\!\!\!\!(a\epsilon_1)_\nabla} \\
 \multicolumn{2}{c}{(a\epsilon_1)_\triangle} & \ddots & \\ & & & & a\epsilon_3 \\\end{pmatrix}
\;\;;\;\;\ve{\cS}\sube =\begin{pmatrix}\red{\frac{aR}{A}-is}\\\vdots\\\red{\frac{aR}{A}-is}\end{pmatrix}\;\;;\;\;\epsilon_{1(3)}:=\frac{(3)A-1}{A}
\end{gather}
\begin{align}
\det\cM\sube =&~2^{A-2}(A+A^{-1})~a^{A-1}
\intertext{After the cluster-coordinate integration, the exponent reads}
\frac{1}{2}\cS\sube ^\intercal\cM\sube ^{-1}\cS\sube -isB\sube =&~\frac{\epsilon_1}{2(A^2+1)}aR^2-\frac{A-1}{A^2+1}iRs-\frac{\epsilon_1}{2(1+A^{-2})}s^2
\\
&~-is\left(\frac{R}{A+1}+\frac{A}{A+1}R^{'}\right)
\end{align}
From the last formula, we identify the coefficients for the quadratic and linear $s$
terms which parametrize the one-dimensional Gaussian integral over $s$.
After this integration, the $\drei{\ve{R}-\ve{R}'}$, as present for the direct 
interaction is transformed into a Gaussian which is non-zero even if the direction
of the relative distance between the centre of mass of the $A$ bodies and the odd man
differs from its parameter value due to the interchange of $\vsp{A}$ and $\vsp{A+1}$:
\begin{equation}
\exp{\left[-\underbrace{\frac{3A^2-1}{2A(A+1)(A^2-1)}a}_{:=\alpha\sube}\ve{R}^2-
\underbrace{\frac{A(A^2+1)}{2(A+1)(A^2-1)}a}_{:=\beta\sube}\ve{R}'^2-
\underbrace{\frac{2A^2}{(A+1)(A^2-1)}a}_{:=\gamma\sube}\ve{R}\cdot\ve{R}'\right]}
\end{equation}
Note that all matrices are chosen to be symmetric in order to employ
$$
\int d^3\ve{v}e^{-\frac{1}{2}\ve{v}^\intercal\cM\ve{v}+\cS^\intercal\cdot\ve{v}}=(2\pi)^{(A-1)/2}~\big(\det\cM\big)^{-1/2}~e^{\frac{1}{2}\cS^\intercal\cM^{-1}\cS}
$$
For the direct term (corresponding to the $\mathbb{1}$ of the antisymmetrizer), only
one (A-1)-dimensional Gaussian integral over the $\vcl{i}$'s has to be evaluated.
The $s$ integral recovers $\drei{\ve{R}-\ve{R}'}$, yielding a constant which is
naturally local (no $\ve{R}'$ dependence). The permutation encodes the probability to
find the particles arranged in $\phi_A$ and $\phi_B(=1)$ with a single particle of
the $A$ system to be present at a distance $\ve{R}$ from its centre of mass while the
$B$ particle resides ``within'' $A$ in a non-local structure. The form of this
structure results from the Gaussian integration over the quadratic $s$ terms.

\subsection{Partial-wave projection}
All complications related to the projection of Eq.\eqref{eq.rgm2} into partial waves
arise even in the absence of an interaction and shall be described below.

\begin{gather}
\left(\hat{T}_{\ve{R}}-E\right)\bra\phi_A\hl\hat{A}\left[\phi_A\psi\right]
\ket=0\\
\left(\hat{T}_{\ve{R}}-E\right)
\int d^3\ve{R}'\left\lbrace\drei{\ve{R}-\ve{R}'}+c\sube e^{-\alpha_k\ve{R}^2-\beta_k\ve{R}'^2-\gamma_k\ve{R}\cdot\ve{R}'}\right\rbrace\psi(\ve{R}')
\intertext{which in virtue of}
c\sube:=-\frac{A}{A+1}\left(\pi(A-1)\right)^{-3/2}\;\;;\;\;\psi(\ve{R})=R^{-1}\sum_{lm}\psi_{lm}(R)Y_{lm}(\hat{\ve{R}})\\
\hat{T}^l_R=\frac{\hbar}{2}\underbrace{\frac{(A+1)}{mA}}_{=\mu^{-1}_\text{\tiny red}}\Big(-\frac{d^2}{dR^2}+\frac{l(l+1)}{R^2}\Big)\;\;;\;\;R\int d^2\hat{\ve{R}}~Y^*_{l'm'}(\hat{\ve{R}})\to(\ldots)
\intertext{yields}
\left(\hat{T}^l_{R}-E\right)\psi_{lm}(R)\\
+c\sube\int dR'~R'e^{-\beta\sube R'^2}\left(\hat{T}^l_{R}-E\right)\left[Re^{-\alpha\sube R^2}\int d^2~(\hat{\ve{R}},\hat{\ve{R'}})Y^*_{l'm'}(\hat{\ve{R}})e^{-\gamma\sube\ve{R}\cdot\ve{R}'}Y_{lm}(\hat{\ve{R}'})\right]\psi_{lm}(R')=0\nonumber
\intertext{in which we substitute}
e^{-\gamma\sube\ve{R}\cdot\ve{R}'}=e^{ii\gamma\sube\ve{R}\cdot\ve{R}'}=4\pi\sum_{LM}i^Lj_L(i\gamma\sube RR')Y^*_{LM}(\hat{\ve{R}})Y_{LM}(\hat{\ve{R}}')
\intertext{to arrive at}
\left(\hat{T}^l_{R}-E\right)\psi_{lm}(R)+4\pi\,i^l\,c\sube\int dR'~R'e^{-\beta\sube R'^2}\left(\hat{T}^l_{R}-E\right)\left[Re^{-\alpha\sube R^2}j_l(i\gamma\sube RR')\right]\psi_{lm}(R')=0\;\;.\nonumber\\
\end{gather}
Enter the interaction potential does not alter this basic structure but adds terms of
the same type:
\begin{gather}
\left(\hat{T}^l_{R}-E\right)\psi_{lm}(R)+\sum_kc_k\int dR'~R'e^{-\beta_k R'^2}
\hat{O}_k(R)
\left[Re^{-\alpha_k R^2}j_l(i\gamma_k RR')\right]\psi_{lm}(R')=0
\intertext{with}
c\sube=-4\pi i^l~\frac{A}{A+1}\left(\pi(A-1)\right)^{-3/2}\;\;\;\;\text{and}\;\;\;\;\hat{O}\sube=\left(\hat{T}^l_{R}-E\right)
\end{gather} 
\newpage
\subsection{Direct- and exchange interaction}
Besides the interaction derived above is a consequence of the identity of 
particles and the ensuing demand for an antisymmetric spatial wave function.
This effective potential presents a deviation from a free motion between
the $A$-body core and a particle even if the particle-particle interaction is zero.
For specific non-zero two- and three-particle interactions we derive the contributions
affecting the cluster-particle relative potential below. Specifically,
we calculate the second term of Eq.~\eqref{eq.rgm2} employing the interaction
Eq.\eqref{eq.v2} and Eq.\eqref{eq.v3}.

For a Gaussian representation of the contact interactions,
\[\ddrei{\ve{x}}=e^{-\frac{\Lambda}{4}\ve{x}^2}\;\;,\]
and $\hat{P}\in\big\lbrace\mathbb{1},\hat{P}_{A,A+1}\big\rbrace$, one arrives at
\begin{align}
\bra\phi_A\hl\hat{V}_\text{\tiny A,A+1}\hl\hat{P}\left[\phi_A\psi\right]\ket=&~
~\int d^3\ve{R}'~\sum\limits_{k=1}^7A_k\left(
\int\limits_{\mathbb{R}}\ldots\int\limits_{\mathbb{R}}d(\cl{1,$\ldots$,A-1},s)~
e^{-\frac{1}{2}\vcl{}^\intercal\cM_k\vcl{}+\ve{\cS}^\intercal_k\cdot\vcl{}-isB_k}
\right)^{\vec{3} }~\blue{\times\psi(\ve{R}')}\;\;\nonumber\\
=&~
\int d^3\ve{R}'~\sum\limits_{k=1}^7A_k\left(
\sqrt{\frac{(2\pi)^{A-1}}{\det\cM_k}}\int\limits_{\mathbb{R}}ds~
e^{\frac{1}{2}\ve{\cS}_k^\intercal\cM^{-1}_k\ve{\cS}_k-isB_k}
\right)^{\vec{3} }~\blue{\times\psi(\ve{R}')}
\intertext{writing the exponent in the form 
$\frac{1}{2}\ve{s}^\intercal\cM_{s,k}\ve{s}+\ve{B}_{s,k}^\intercal\cdot\ve{s}$
translates the $s$ integral into}
=&~\int d^3\ve{R}'~\sum\limits_{k=1}^7A_k\left(
\sqrt{\frac{(2\pi)^{A-1}}{\det\cM_k}}\cdot\sqrt{\frac{2\pi}{\det\cM_{s,k}}}
\right)^3\times
e^{-\alpha_k\ve{R}^2-\beta_k\ve{R}'^2-\gamma_k\ve{R}\cdot\ve{R}'}
\blue{\times\psi(\ve{R}')}\nonumber\\
=&~\sum\limits_{k=1}^7A_k\left(
\frac{(2\pi)^A}{\det\cM_k~\det\cM_{s,k}}
\right)^{\frac{3}{2}}~\int d^3\ve{R}'~
e^{-\alpha_k\ve{R}^2-\beta_k\ve{R}'^2-\gamma_k\ve{R}\cdot\ve{R}'}
\blue{\times\psi(\ve{R}')}
\la{eq.expot}
\end{align}
\ie, a structure which is analogous to Eq.\eqref{eq.exkernel}. 
In order to obtain the coefficients,
interacting pairs
and triples are organized in seven terms, each of which corresponding to the same
exponent. Specifically, we found the following arrangement convenient (details in
Sec.\ref{sec.int}).
\begin{align}
C_0(\Lambda)\sum\limits_{\red{i\leq A}}\ddrei{\vsp{i}-\vsp{A+1}}
=&~\left\lbrace
\sum\limits_{\red{i<A}}\drei{\vcl{\red{i}}-R}+\drei{\scriptstyle -\sum\limits_{i<A}\vcl{i}-R}
\right\rbrace\cdot C_0(\Lambda)\la{eq.v21}
\\
=&~\left\lbrace\,
(A-1)\,\ddrei{\vcl{\red{1}}-R}+1\cdot\ddrei{\scriptstyle -\sum\limits_{i<A}\vcl{i}-R}
\right\rbrace\cdot C_0(\Lambda)\la{eq.v22}
\\
D_1(\Lambda)\sum^A_{\red{i< j\atop \text{\tiny cyc}}}
\ddrei{\vsp{i}-\vsp{A+1}}\ddrei{\vsp{j}-\vsp{A+1}}=&~
\left\lbrace{A-1\choose 2}
\ddrei{\vsp{1}-\vsp{A+1}}\ddrei{\vsp{2}-\vsp{A+1}}\right.\la{eq.v31}
\\
+&~\left. (A-1)\,
\ddrei{\vsp{A}-\vsp{A+1}}\ddrei{\vsp{1}-\vsp{A+1}}\right\rbrace
\cdot D_1(\Lambda)\la{eq.v32}
\\
D_1(\Lambda)\sum\limits_{\red{i< j\atop \text{\tiny cyc}}}
\ddrei{\vsp{i}-\vsp{A+1}}\ddrei{\vsp{j}-\vsp{j}}=&~
\left\lbrace{A-1\choose 2}
\ddrei{\vsp{1}-\vsp{A+1}}\ddrei{\vsp{1}-\vsp{2}}\right.\la{eq.v33}
\\
+&~(A-1)\ddrei{\vsp{A}-\vsp{A+1}}\ddrei{\vsp{A}-\vsp{1}}\la{eq.v34}
\\
+&~\left.(A-1)\ddrei{\vsp{1}-\vsp{A+1}}\ddrei{\vsp{1}-\vsp{A}}\right\rbrace\cdot D_1(\Lambda)\;\;\;,\la{eq.v35}
\end{align}
and written as a function of cluster, \ie, integration variables:
\begin{eqnarray}
\hat{V}_{A,A+1}=&~C_0(\Lambda)\cdot&
\left\lbrace\,(A-1)\,\ddrei{\vcl{\red{1}}-R}+1\cdot
\ddrei{\scriptstyle \sum\limits_{i<A}\vcl{i}+R}\right\rbrace_{\blue{k=1,2}}\nonumber\\
+&~
D_1(\Lambda)\cdot&\left\lbrace{A-1\choose 2}\left(\ddrei{\vcl{1}-R}\ddrei{\vcl{2}-R}+
\ddrei{\vcl{1}-R}\ddrei{\vcl{1}-\vcl{2}}\right)_{\blue{k=3,5}}\right.\nonumber\\
&&+(A-1)\left(\ddrei{\sum\limits_{i<A}\vcl{i}+R}\ddrei{\vcl{1}-R}+\ddrei{\sum\limits_{i<A}\vcl{i}+R}\ddrei{\sum\limits_{i<A}\vcl{i}+\vcl{1}}\right.\nonumber\\
&&~\;\;\;\;\;\;\;\;\;\;\;\;\;+\left.\left.
\ddrei{\vcl{1}-R}\ddrei{\sum\limits_{i<A}\vcl{i}+\vcl{1}}\right)_{\blue{k=4,6,7}}\right\rbrace\;\;.
\end{eqnarray}
From these 7 terms, the matrices and vectors which parametrize the integral in
Eq.\eqref{eq.expot} are read off (the blue subscripts refer to the parameter set
which results from the respective term in the brackets):
%
\newpage
%
\arraycolsep=2.4pt\def\arraystretch{1.25}
\begin{align*}
\cM_1 =&
\begin{pmatrix}
a\green{\epsilon_3}+\frac{\Lambda^2}{2} & & & \\ 
& a\green{\epsilon_3} & &\multicolumn{2}{l}{\!\!\!\!(a\green{\epsilon_1})_\nabla} \\
 \multicolumn{2}{c}{(a\green{\epsilon_1})_\triangle} & \ddots & \\
  & & & & a\green{\epsilon_3} \\
\end{pmatrix}
&
\ve{\cS}_1 =&\begin{pmatrix}\red{\frac{aR}{A}-is}&+\frac{\Lambda^2}{2}R\\\red{\frac{aR}{A}-is}&\\\vdots&\\\red{\frac{aR}{A}-is}&\end{pmatrix}
&
B'_1=&-\frac{i}{s}\left(\red{\frac{a\epsilon_1}{2}}+\frac{\Lambda^2}{4}\right)R^2
\\
\cM_2 =&
\begin{pmatrix}
a\green{\epsilon_3}+\frac{\Lambda^2}{2} & & & \\ 
& a\green{\epsilon_3}+\frac{\Lambda^2}{2} & &\multicolumn{2}{l}{\!\!\!\!(a\green{\epsilon_1}+\frac{\Lambda^2}{2})_\nabla} \\
 \multicolumn{2}{c}{(a\green{\epsilon_1}+\frac{\Lambda^2}{2})_\triangle} & \ddots & \\
  & & & & a\green{\epsilon_3}+\frac{\Lambda^2}{2} \\
\end{pmatrix}
&
\ve{\cS}_2 =&\begin{pmatrix}\red{\frac{aR}{A}-is}-\frac{\Lambda^2}{2}R\\ \vdots\\
\red{\frac{aR}{A}-is}-\frac{\Lambda^2}{2}R\end{pmatrix}
&
B'_2=&-\frac{i}{s}\left(\red{\frac{a\epsilon_1}{2}}+\frac{\Lambda^2}{4}\right)R^2
\\
\cM_3 =&
\begin{pmatrix}
a\green{\epsilon_3}+\frac{\Lambda^2}{2} & & & &\\ 
& a\green{\epsilon_3}+\frac{\Lambda^2}{2} & &\multicolumn{2}{l}{\!\!\!\!(a\green{\epsilon_1})_\nabla} \\
&& a\green{\epsilon_3} &&\\
 \multicolumn{3}{c}{(a\green{\epsilon_1})_\triangle} & \ddots & \\
  & & & & a\green{\epsilon_3}
\end{pmatrix}
&
\ve{\cS}_3 =&\begin{pmatrix}\red{\frac{aR}{A}-is}&+\frac{\Lambda^2}{2}R\\\red{\frac{aR}{A}-is}&+\frac{\Lambda^2}{2}R \\ \red{\frac{aR}{A}-is}&\\\vdots&\\
\red{\frac{aR}{A}-is}\end{pmatrix}
&
B'_3=&-\frac{i}{s}\left(\red{\frac{a\epsilon_1}{2}}+\frac{\Lambda^2}{2}\right)R^2
\\
\cM_4 =&
\begin{pmatrix}
a\green{\epsilon_3}+\Lambda^2 & & & \\ 
& a\green{\epsilon_3}+\frac{\Lambda^2}{2} & &\multicolumn{2}{l}{\!\!\!\!(a\green{\epsilon_1}+\frac{\Lambda^2}{2})_\nabla} \\
 \multicolumn{2}{c}{(a\green{\epsilon_1}+\frac{\Lambda^2}{2})_\triangle} & \ddots & \\
  & & & & a\green{\epsilon_3}+\frac{\Lambda^2}{2} \\
\end{pmatrix}
&
\ve{\cS}_4 =&\begin{pmatrix}\red{\frac{aR}{A}-is}&\\\red{\frac{aR}{A}-is}&-\frac{\Lambda^2}{2}R \\
\vdots&\\ \red{\frac{aR}{A}-is}&-\frac{\Lambda^2}{2}R\end{pmatrix}
&
B'_4=&-\frac{i}{s}\left(\red{\frac{a\epsilon_1}{2}}+\frac{\Lambda^2}{2}\right)R^2
\\
\cM_5 =&
\begin{pmatrix}
a\green{\epsilon_3}+\Lambda^2 & a\green{\epsilon_1}-\frac{\Lambda^2}{2} & & &\\ 
a\green{\epsilon_1}-\frac{\Lambda^2}{2} & a\green{\epsilon_3}+\frac{\Lambda^2}{2} & &\multicolumn{2}{l}{\!\!\!\!(a\green{\epsilon_1})_\nabla} \\
&& a\green{\epsilon_3} &&\\
 \multicolumn{3}{c}{(a\green{\epsilon_1})_\triangle} & \ddots & \\
  & & & & a\green{\epsilon_3} \\
\end{pmatrix}
&
\ve{\cS}_5 =&\begin{pmatrix}\red{\frac{aR}{A}-is}&+\frac{\Lambda^2}{2}R\\\red{\frac{aR}{A}-is}& \\
\vdots&\\ \red{\frac{aR}{A}-is}&\end{pmatrix}
&
B'_4=&-\frac{i}{s}\left(\red{\frac{a\epsilon_1}{2}}+\frac{\Lambda^2}{4}\right)R^2
\\
\cM_6 =&
\begin{pmatrix}
a\green{\epsilon_3}+\frac{5}{2}\Lambda^2 & a\green{\epsilon_1}+\frac{3}{2}\Lambda^2 & 
\multicolumn{2}{l}{\ldots} &a\green{\epsilon_1}+\frac{3}{2}\Lambda^2 \\ 
a\green{\epsilon_1}+\frac{3}{2}\Lambda^2 & a\green{\epsilon_3}+\Lambda^2 & & &\\
\vdots&& \ddots\;\;\; &\multicolumn{2}{c}{\!\!\!\!(a\green{\epsilon_1}+\Lambda^2)_\nabla}\\
 &\multicolumn{2}{c}{(a\green{\epsilon_1}+\Lambda^2)_\triangle} &  & \\
  a\green{\epsilon_1}+\frac{3}{2}\Lambda^2& & & & a\green{\epsilon_3}+\Lambda^2 \\
\end{pmatrix}
&
\ve{\cS}_6 =&\begin{pmatrix}\red{\frac{aR}{A}-is}-\frac{\Lambda^2}{2}R\\ \vdots\\
\red{\frac{aR}{A}-is}-\frac{\Lambda^2}{2}R\end{pmatrix}
&
B'_6=&-\frac{i}{s}\left(\red{\frac{a\epsilon_1}{2}}+\frac{\Lambda^2}{4}\right)R^2
\\
\cM_7 =&
\begin{pmatrix}
a\green{\epsilon_3}+\frac{5}{2}\Lambda^2 & a\green{\epsilon_1}+\Lambda^2 & 
\multicolumn{2}{l}{\ldots} &a\green{\epsilon_1}+\Lambda^2 \\ 
a\green{\epsilon_1}+\Lambda^2 & a\green{\epsilon_3}+\frac{\Lambda^2}{2} & & &\\
\vdots&& \ddots\;\;\; &\multicolumn{2}{c}{\!\!\!\!(a\green{\epsilon_1}+\frac{\Lambda^2}{2})_\nabla}\\
 &\multicolumn{2}{c}{(a\green{\epsilon_1}+\frac{\Lambda^2}{2})_\triangle} &  & \\
  a\green{\epsilon_1}+\Lambda^2& & & & a\green{\epsilon_3}+\frac{\Lambda^2}{2} \\
\end{pmatrix}
&
\ve{\cS}_7 =&\begin{pmatrix}\red{\frac{aR}{A}-is}&+\frac{\Lambda^2}{2}R\\\red{\frac{aR}{A}-is}&\\ \vdots&\\
\red{\frac{aR}{A}-is}&\end{pmatrix}
&
B'_7=&-\frac{i}{s}\left(\red{\frac{a\epsilon_1}{2}}+\frac{\Lambda^2}{4}\right)R^2
\\
&&&&B_k=&B'_k+\green{\underbrace{\left(\frac{R}{A+1}+\frac{A}{A+1}R'\right)}_{\stackrel{\hat{P}=\mathbb{1}}{\to}R-R'}}\\
&\epsilon_1=
  \begin{cases}
   2             \\
   \frac{A-1}{A} 
  \end{cases}
\epsilon_3=
  \begin{cases}
   4        & \text{if } \hat{P}=\mathbb{1} \\
   \frac{3A-1}{A}        & \text{if } \hat{P}=\hat{P}_{A,A+1}
  \end{cases}  
  &&\multispan3{$\red{\text{red}}~=~0\;\;\text{if}\;\;\hat{P}=\mathbb{1}$\hfil}\\
\end{align*}
\newpage
%
These matrices enable the derivation of
\begin{align}
\bra\phi_A\hl\hat{V}_\text{\tiny A,A+1}\hl\hat{P}\left[\phi_A\psi\right]\ket=&~
\sum_{k=1}^7A_k\left(\frac{(2\pi)^{A-1}}{\det\cM_k}\right)^{\frac{3}{2}}
e^{-\alpha_k\ve{R}^2}\psi(\ve{R})\nonumber\\
&~\red{-}\sum_{k=8}^{14}\frac{A}{A+1}(2\pi)^{-3}A_k
\left(\frac{(2\pi)^A}{\det\cM_k\cdot\det\cM_{s,k}}\right)^{\frac{3}{2}}
\int d^3\ve{R'}e^{-\alpha_k\ve{R}^2-\beta_k\ve{R'}^2-\gamma_k\ve{R}\cdot\ve{R'}}\psi(\ve{R'})
\nonumber\\
=&~(2\pi)^{\frac{3}{2}(A-1)}\sum_{k=1}^7A_k\int d^3\ve{R'}\left[
\left(\det\cM_k^{\text{dir}}\right)^{-\frac{3}{2}}\drei{\ve{R}-\ve{R'}}
e^{-\alpha_k^{\text{dir}}\ve{R'}^2}
\right.\nonumber\\
\red{-}&~\left.
\frac{A}{A+1}(2\pi)^{-\frac{3}{2}}\left(\det\cM_k^{\text{ex}}\cdot
\det\cM_{s,k}\right)^{-\frac{3}{2}}
e^{-\alpha_k^{\text{ex}}\ve{R}^2-\beta_k^{\text{ex}}\ve{R'}^2-\gamma_k^{\text{ex}}\ve{R}\cdot\ve{R'}}
\right]\psi(\ve{R'})
\end{align}
%
\subsubsection{Interacting notes}\la{sec.int}
We represent the microscopic, \ie, particle-particle interaction with properly
renormalized two- and three-body contacts: Particles interact only if two or three
of them occupy the same point in space. The intensity of the interaction depends
on whether two or three particles collide:
\[
\hat{V}=C_0(\Lambda)\sum_{i<j}^X\ddrei{\vsp{i}-\vsp{j}}+
D_1(\Lambda)\sum^X_{i<j<k\atop\text{\tiny cyclic}}\ddrei{\vsp{i}-\vsp{j}}\,
\ddrei{\vsp{j}-\vsp{k}}\;\;.
\la{eq.int}\]
For $X\leq A$, the interaction yields stable, self-bound $X$-body ground states with
totally symmetric spatial wave functions, because we investigate systems whose
internal space is precisely $A$-dimensional.
For $X=A+1$, we isolate the interaction of particle $A+1$ with the $A$-body fragment:
\begin{align*}
\hat{V}=\hat{V}_A+\green{\hat{V}_{A,A+1}}=&~
\sum_{i<j}^A\delta_{i,j}+\green{\sum_{i}^{A-1}\delta_{i,A+1}+\delta_{A,A+1}}\\
&+\sum_{i<j<k\atop\text{\tiny cyc}}^A\delta_{i,j,k}+\green{\sum_{i<j\atop\text{\tiny cyc}}^{A-1}\delta_{i,j,A+1}+\sum_{i\atop\text{\tiny cyc}}^{A-1}\delta_{i,A,A+1}}\\
\hat{V}_{A,A+1}\stackrel{\scriptscriptstyle \langle\ldots\rangle}{\sim}&~
\underbrace{(A-1)\,\delta_{1,A+1}+1\cdot\delta_{A,A+1}}_{\eqref{eq.v21},\eqref{eq.v22}}+\underbrace{2\cdot{A-1\choose 2}
\delta_{1,2,A+1}}_{\Rightarrow\eqref{eq.v31}+\eqref{eq.v33}}+\underbrace{3\cdot(A-1)\,\delta_{1,A,A+1}}_{\eqref{eq.v32}+\eqref{eq.v34}+\eqref{eq.v35}}\;\;.
\la{eq.clint}\end{align*}
Matrix elements in the $\phi_A\psi$ basis as defined in Eqs.\eqref{eq.wfktphi}
and~\eqref{eq.wfktpsi} are identical
($\stackrel{\scriptscriptstyle \langle\ldots\rangle}{\sim}$) for left and right-hand-side
operators.
\newpage
%\section*{Appendix: $F[2]B[2]$-RGM equation }
\openup 2ex
\begin{gather*}
\intertext{\Large\centering Assumption of a tightly-bound, spatially-symmetric $A$-body core:}
\phi_A:=e^{-\frac{a}{2}\sum_{i=1}^{\red{A}}\vcl{i}^2}=e^{-a\sum^{\red{A-1}}\vcl{i}^2-a\sum^{\red{A-1}}_{\red{i<j}}\vcl{i}\cdot\vcl{j}}
\intertext{\Large\centering The contact theory:}
\hat{V}=C_0(\Lambda)\sum_{i<j}^X\ddrei{\vsp{i}-\vsp{j}}+
D_1(\Lambda)\sum^X_{i<j<k\atop\text{\tiny cyclic}}\ddrei{\vsp{i}-\vsp{j}}\,
\ddrei{\vsp{j}-\vsp{k}}\\
\intertext{\Large\centering Enter indistinguishability:}
\hat{A}=\mathbb{1}-\hat{P}(\vsp{A}\leftrightarrow\vsp{A+1})
\intertext{\Large\centering ``Freeze'' the core and extremize the action \wrt~the relative motion between
the one particle outside of the core:}
%
\left(\hat{T}_{\ve{R}}-E_\text{rel}+\mathbb{N}^{-1}\bra\phi_A\hl\hat{V}\hl\phi_A\ket\right)\chi(\ve{R})\\
-\mathbb{N}^{-1}\int d\ve{R}'\left[\bra\phi_A\hl\left(\hat{T}_{\ve{R}}-E_\text{rel}+\hat{V}\right)\hat{P}\lbrace\hl\phi_A\ket\ddrei{\ve{R}-\ve{R}'}\right]\chi(\ve{R}')=0\\
\intertext{\Large\centering $\hat{V}=C_\Lambda\sum\limits_{i<A}\ddrei{\vsp{i}-\vsp{A+1}}$ and $\hat{P}=0$ \ie~direct 2-body contribution only:}
\left(\hat{T}_{\ve{R}}-E_\text{rel}+C_\Lambda(A-1)~8\left(4+\left(\frac{A-1}{A}\right)a^{-1}\Lambda^2\right)^{-3/2}\cdot e^{-\frac{\Lambda^2}{4+A^{-1}(A-1)a^{-1}\Lambda^2}\ve{R}^2}\right)\chi(\ve{R})=0\\
\downarrow\\
-\frac{\hbar}{2m}\frac{A+1}{A}\partial_R^2+\left(\frac{\hbar}{2m}\frac{A+1}{A}\frac{l(l+1)}{R^2}+\frac{C_\Lambda(A-1)~8}{\left(4+\left(\frac{A-1}{A}\right)a^{-1}\Lambda^2\right)^{3/2}}\cdot e^{-\frac{\Lambda^2}{4+A^{-1}(A-1)a^{-1}\Lambda^2}\ve{R}^2}\right)\psi_{lm}(R)=0\\
\Leftrightarrow-\frac{\hbar}{2m}\partial_R^2+\left(\frac{\hbar}{2m}\frac{l(l+1)}{R^2}+\frac{8~C_\Lambda A(A-1)(A+1)^{-1}}{\left(4+\left(\frac{A-1}{A}\right)a^{-1}\Lambda^2\right)^{3/2}}\cdot e^{-\frac{\Lambda^2}{4+A^{-1}(A-1)a^{-1}\Lambda^2}\ve{R}^2}\right)\psi_{lm}(R)=0\\
\intertext{\scalebox{2}{\Large\centering \Large\centering with}}
E_\text{relative}=\epsilon_A-E_\text{total}\;\;\;\text{and}\;\;\;\mathbb{N}=\bra\phi_A\hl\phi_A\ket
\end{gather*}
\newpage
\begin{itemize}\itemsep25pt
\item[]\Large\centering Assumption of a tightly-bound, spatially-symmetric $A$-body core:\vspace{1.cm}\\
\scalebox{1.5}{$\phi_A:=e^{-\frac{a}{2}\sum_{i=1}^{\red{A}}\vcl{i}^2}=e^{-a\sum^{\red{A-1}}\vcl{i}^2-a\sum^{\red{A-1}}_{\red{i<j}}\vcl{i}\cdot\vcl{j}}$}

\item[] \Large\centering The contact theory:\vspace{1.cm}\\
\scalebox{1.2}{$\hat{V}=C_0(\Lambda)\sum_{i<j}^X\ddrei{\vsp{i}-\vsp{j}}+
D_1(\Lambda)\sum^X_{i<j<k\atop\text{\tiny cyclic}}\ddrei{\vsp{i}-\vsp{j}}\,
\ddrei{\vsp{j}-\vsp{k}}$}\\

\item[] \Large\centering Enter indistinguishability:\vspace{1.cm}\\
\scalebox{1.5}{$\hat{A}=\mathbb{1}-\hat{P}(\vsp{A}\leftrightarrow\vsp{A+1})$}

\item[] \Large\centering ``Freeze'' the core and extremize the action \wrt~the relative motion between
the one particle outside of the core:\vspace{1.cm}\\
\scalebox{1.5}{$\left(\hat{T}_{\ve{R}}-E_\text{rel}+\mathbb{N}^{-1}\bra\phi_A\hl\hat{V}\hl\phi_A\ket\right)\chi(\ve{R})$}\\
\scalebox{1.15}{$-\mathbb{N}^{-1}\int d\ve{R}'\left[\bra\phi_A\hl\left(\hat{T}_{\ve{R}}-E_\text{rel}+\hat{V}\right)\hat{P}\lbrace\hl\phi_A\ket\ddrei{\ve{R}-\ve{R}'}\right]\chi(\ve{R}')=0$}

\item[] \Large\centering $\hat{V}=C_\Lambda\sum\limits_{i<A}\ddrei{\vsp{i}-\vsp{A+1}}$ and $\hat{P}=0$ \ie~direct 2-body contribution:\vspace{1.2cm}\\
\scalebox{1.0}{$\left(\hat{T}_{\ve{R}}-E_\text{rel}+C_\Lambda(A-1)~8\left(4+\left(\frac{A-1}{A}\right)a^{-1}\Lambda^2\right)^{-3/2}\cdot e^{-\frac{\Lambda^2}{4+A^{-1}(A-1)a^{-1}\Lambda^2}\ve{R}^2}\right)\chi(\ve{R})=0$}\\
\scalebox{1.15}{$\downarrow$}\\
\scalebox{1.15}{$-\frac{\hbar}{2m}\frac{A+1}{A}\partial_R^2+\left(\frac{\hbar}{2m}\frac{A+1}{A}\frac{l(l+1)}{R^2}+\frac{C_\Lambda(A-1)~8}{\left(4+\left(\frac{A-1}{A}\right)a^{-1}\Lambda^2\right)^{3/2}}\cdot e^{-\frac{\Lambda^2}{4+A^{-1}(A-1)a^{-1}\Lambda^2}\ve{R}^2}\right)\psi_{lm}(R)=0$}\\
\scalebox{1.5}{$\Leftrightarrow-\frac{\hbar}{2m}\partial_R^2+\left(\frac{\hbar}{2m}\frac{l(l+1)}{R^2}+\frac{8~C_\Lambda A(A-1)(A+1)^{-1}}{\left(4+\left(\frac{A-1}{A}\right)a^{-1}\Lambda^2\right)^{3/2}}\cdot e^{-\frac{\Lambda^2}{4+A^{-1}(A-1)a^{-1}\Lambda^2}\ve{R}^2}\right)\psi_{lm}(R)=0$}

\item[]\Large\centering \Large\centering with\vspace{.2cm}\\
\scalebox{1.5}{$E_\text{relative}=\epsilon_A-E_\text{total}\;\;\;\text{and}\;\;\;\mathbb{N}=\bra\phi_A\hl\phi_A\ket$}
\end{itemize}
\newpage
\bibliographystyle{unsrt}
\bibliography{Thebibliography.bib}

\end{document}
