\documentclass[onecolumn,preprint,superscriptaddress,nofootinbib,notitlepage,10pt,linenumbers]{revtex4-1}

\usepackage{graphicx} 
\usepackage{dashrule}
\usepackage{dcolumn} 
\usepackage{longtable} 
\usepackage{mathtools}
\usepackage{amssymb} 
\usepackage{bbold}
\usepackage{amsmath} 
\usepackage{amsfonts} 
\usepackage{wasysym}
\usepackage{slashed} 
\usepackage[dvipsnames]{xcolor} 
\usepackage{soul}
\usepackage{rotating}
\usepackage{paracol}
%\usepackage{pdflscape}
%=============================================================================
\def\cM{{\widehat{M}}}
\newcommand{\sube}{_\text{\tiny EX}}
\newcommand{\subd}{_\text{\tiny D}}
\def\cS{{\mathcal S}}
\newcommand*{\mprime}{^{\prime}\mkern-1.2mu}
\newcommand*{\mdprime}{^{\prime\prime}\mkern-1.2mu}
\newcommand*{\mtprime}{^{\prime\prime\prime}\mkern-1.2mu}
\newcommand{\red}[1]{\textcolor{red}{#1}}
\newcommand{\green}[1]{\textcolor{green}{#1}} 
\newcommand{\gray}[1]{\textcolor{gray}{#1}} 
\newcommand{\blue}[1]{\textcolor{blue}{#1}} 
\newcommand{\bs}[1]{\boldsymbol{#1}} 
\newcommand{\nopieft}{\mbox{$\slashed{\pi}$EFT}} 
\newcommand{\eftnopi}{\mbox{EFT($\slashed{\pi}$) }} 
\newcommand{\Lag}{{\cal L}}
\newcommand{\la}{\label} 
\newcommand{\be}{\begin{equation}} 
\newcommand{\ee}{\end{equation}} 
\newcommand{\vcl}[1]{\ensuremath{\bar{\boldsymbol{r}}_\text{\tiny #1}}}
\newcommand{\toinf}{\ensuremath{\stackrel{\text{\scalebox{0.6}{$\Lambda\to\infty$}}}{\longrightarrow}}}
\newcommand{\cl}[1]{\ensuremath{\bar{r}_\text{\tiny #1}}}
\newcommand{\vsp}[1]{\ensuremath{\boldsymbol{r}}_\text{\tiny #1}}
\newcommand{\rvec}{{\bs{r}}} 
\newcommand{\xvec}{{\bs{x}}} 
\newcommand{\sgmvec}{\ensuremath{\boldsymbol{\sigma}}} 
\newcommand{\tauvec}{\ensuremath{\boldsymbol{\tau}}}
\newcommand{\fm}{\ensuremath{\,\text{fm}^{-1}}} 
\newcommand{\half}{\frac{1}{2}}
\newcommand{\eg}{\textit{e.g.}\;}
\newcommand{\ie}{\textit{i.e.}\;}
\newcommand{\wrt}{\textit{wrt.}\;}
\newcommand{\etc}{\textit{etc.}\;}
\newcommand{\ve}[1]{\ensuremath{\boldsymbol{#1}}}
\newcommand{\ddrei}[1]{\delta_{\tiny \Lambda}^{(3)}\!\big(#1\big)}
\newcommand{\abb}{\ensuremath{2\!+\!1^{(A-1)}}}
\newcommand{\coup}[3]{\left[\,#1\,\otimes\,#2\,\right]^{#3}}
\newcommand{\threej}[6]{ \begin{pmatrix}
   #1 & #2 & #3 \\
   #4 & #5 & #6 
  \end{pmatrix}}
\newcommand{\lam}[1]{\mbox{\ensuremath{\Lambda=#1\,\text{fm}^{-1}}}}   
\newcommand{\figref}[1]{fig.~\ref{#1}}
\newcommand{\tabref}[1]{table~\ref{#1}}
\newcommand{\cc}{\ensuremath C_0(\Lambda)}
\newcommand{\dd}{\ensuremath D_1(\Lambda)}
\newcommand{\abab}{\ensuremath $(ab)\!\!:\!\!(ab)$}
\newcommand{\abca}{\ensuremath $(ab)\wasytherefore(ca)$}

\begin{document}

\section{dimer-dimer scattering}
In the single-channel approximation, the resonating-group equation assumes the non-local form
\be\la{eq.rgm}
(\hat{T}-E)~\chi(\ve{r})+\mathcal{V}^{(1)}(\ve{r})~\chi(\ve{r})+
\int d^{(3)}\ve{r}\mprime~\mathcal{V}^{(2)}(\ve{r},\ve{r}\mprime,E)~\chi(\ve{r}\mprime)=0
\ee
with the radial coordinates denoting the spatial separation between the two fragments. If these fragments are
two-body $S$-wave bound states comprised of equal-mass fermions, the effective potentials which derive from a
zero-range fermion-fermion interaction are given for a two- and three-species system. We denote the former
as \abab~(scale invariant), and the latter \abca(discretely scale invariant, Thomas
collapse of $(abc)$). The characteristic three-body scale in an \abca~system flows into the effective
dimer-dimer potentials, while in the absence of such a scale in the zero-range two-body limit, the effective
potentials are parametrized by the dimer, \ie, a two-body observable, only.

In detail, 

\begin{description}
	\item[\abab]
\be\la{eq.local.abab}
\mathcal{V}^{(1)}_{\text{\tiny\abab}}(\ve{r})=
2\,C_0(\lambda)\cdot\left(\frac{2\alpha}{2\alpha+\lambda}\right)^{3/2}\cdot
 e^{-\frac{2\alpha\lambda}{2\alpha+\lambda}\ve{r}^2}\;\;,
\ee
\be\la{eq.nonlocal.abab}
\mathcal{V}^{(2)}_{\text{\tiny\abab}}(\ve{r},\ve{r}\mprime,E)=
~8~\alpha^{3/2}\cdot e^{-\alpha\ve{r}\mprime^2}\cdot
\left[\frac{\hbar^2}{2\mu}~(4\alpha^2\ve{r}^2-2\alpha)
\cdot e^{-\alpha\ve{r}^2}+
E\cdot e^{-\alpha\ve{r}^2}
-2~C_0(\lambda)\cdot\left(\frac{2\alpha}{2\alpha+\lambda}\right)^{3/2}\cdot
 e^{-\alpha\cdot\frac{2\alpha+3\lambda}{2\alpha+\lambda}\ve{r}^2}\right]\;\;.
\ee

\item[\abca]
\begin{eqnarray}\la{eq.local.abca}
\mathcal{V}^{(1)}_{\text{\tiny\abca}}(\ve{r})&=&
3\cdot C_0(\lambda)\cdot\left(\frac{2\alpha}{2\alpha+\lambda}\right)^{3/2}\cdot
 e^{-\frac{2\alpha\lambda}{2\alpha+\lambda}\ve{r}^2}\\
&&+D_0(\lambda)\cdot\left(
\left(\frac{2\alpha}{2\alpha+\lambda}\right)^{3}\cdot e^{-\frac{4\alpha\lambda}{2\alpha+\lambda}\ve{r}^2}+
\left(\frac{2\alpha}{\sqrt{(2\alpha+\lambda)^2+2\alpha\lambda}}\right)^{3}\cdot
 e^{-\frac{4\alpha\lambda(\alpha+\lambda)}{4\alpha^2+6\alpha\lambda+\lambda^2}\ve{r}^2}
\right)
\end{eqnarray}


\begin{eqnarray}\la{eq.nonlocal.abca}
\mathcal{V}^{(2)}_{\text{\tiny\abca}}(\ve{r},\ve{r}\mprime,E)=&~8~\alpha^{3/2}\cdot\Bigg(&e^{-\alpha\ve{r}\mprime^2}\cdot
\left[\frac{\hbar^2}{2\mu}~(4\alpha^2\ve{r}^2-2\alpha)
\cdot e^{-\alpha\ve{r}^2}+
E\cdot e^{-\alpha\ve{r}^2}\right]\\
&&-C_0(\lambda)\cdot
 e^{-(\alpha+\lambda)(\ve{r}^2+\ve{r}\mprime^2)-2\lambda\ve{r}\mprime\cdot\ve{r}}
-2~C_0(\lambda)\cdot\left(\frac{2\alpha}{2\alpha+\lambda}\right)^{3/2}\cdot
 e^{-\alpha\cdot\left(\ve{r}\mprime^2+\frac{2\alpha+3\lambda}{2\alpha+\lambda}\ve{r}^2\right)}\\
 &&-~D_0(\lambda)\cdot\left(\frac{\alpha}{\alpha+\lambda}\right)^{3/2}\cdot
 e^{-\frac{2\alpha^2+4\alpha\lambda+\lambda^2}{2(\alpha+\lambda)}(\ve{r}^2+\ve{r}\mprime^2)-\frac{\lambda^2}{\alpha+\lambda}\ve{r}\cdot\ve{r}\mprime}\\
 &&-D_0(\lambda)\cdot\left(\frac{2\alpha(\alpha+\lambda)}{2\alpha^2+3\alpha\lambda+\lambda^2}\right)^{3/2}\cdot
 e^{-\frac{2\alpha^2+5\alpha\lambda+\lambda^2}{2(\alpha+\lambda)}\ve{r}^2-(\alpha+\lambda)\ve{r}\mprime^2-2\lambda\ve{r}\cdot\ve{r}\mprime}\Bigg)
\end{eqnarray}

\end{description}

It is in order to consider the following limits: 
\begin{description}
\item[zero-range or contact limit]$\lambda\gg\alpha$
\item[local approximation]
$\int d^{(3)}\ve{r}\mprime~\mathcal{V}^{(2)}(\ve{r},\ve{r}\mprime,E)~\chi(\ve{r}\mprime)\stackrel{E\to 0}{\approx}
\chi(\ve{r})\cdot v^{(2)}(\ve{r})\cdot\int d^{(3)}\ve{r}\mprime~v^{(2)}(\ve{r}\mprime)$~.
\end{description}

Assuming an unnaturally large dimer scale emergent from a relatively short-ranged fermion-fermion interaction,
the zero-range approximation is justified and the ensuing dimer-dimer potentials read:

\begin{description}
	\item[(zero-range)~\abab]
\be\la{eq.local.abab}
\mathcal{V}^{(1)}_{\text{\tiny\abab}}(\ve{r})=
2\,(2\alpha)^{3/2}~\frac{C_0(\lambda)}{\lambda^{3/2}}\cdot
 e^{-2\alpha\ve{r}^2}\;\;,
\ee
\be\la{eq.nonlocal.abab}
\mathcal{V}^{(2)}_{\text{\tiny\abab}}(\ve{r},\ve{r}\mprime,E)=
~8~\alpha^{3/2}\cdot e^{-\alpha\ve{r}\mprime^2}\cdot
\left[\frac{\hbar^2}{2\mu}~(4\alpha^2\ve{r}^2-2\alpha)
\cdot e^{-\alpha\ve{r}^2}+
E\cdot e^{-\alpha\ve{r}^2}
-2\,(2\alpha)^{3/2}~\frac{C_0(\lambda)}{\lambda^{3/2}}\cdot
 e^{-3\alpha\ve{r}^2}\right]\;\;.
\ee

\item[(zero-range)~\abca]
\begin{eqnarray}\la{eq.local.abca}
\mathcal{V}^{(1)}_{\text{\tiny\abca}}(\ve{r})&=&
3\,(2\alpha)^{3/2}~\frac{C_0(\lambda)}{\lambda^{3/2}}\cdot
 e^{-2\alpha\ve{r}^2}
 +2\,(2\alpha)^{3}~\frac{D_0(\lambda)}{\lambda^{3}}\cdot
 e^{-4\alpha\ve{r}^2}
\end{eqnarray}


\begin{eqnarray}\la{eq.nonlocal.abca}
\mathcal{V}^{(2)}_{\text{\tiny\abca}}(\ve{r},\ve{r}\mprime,E)=&~8~\alpha^{3/2}\cdot\Bigg(&e^{-\alpha\ve{r}\mprime^2}\cdot
\left[\frac{\hbar^2}{2\mu}~(4\alpha^2\ve{r}^2-2\alpha)
\cdot e^{-\alpha\ve{r}^2}+
E\cdot e^{-\alpha\ve{r}^2}\right]\\
&&-C_0(\lambda)\cdot
 e^{-\lambda(\ve{r}+\ve{r}\mprime)^2}
-2~(2\alpha)^{3/2}\,\frac{C_0(\lambda)}{\lambda^{3/2}}\cdot
 e^{-\alpha\ve{r}\mprime^2-3\alpha\ve{r}^2}\\
 &&-\alpha^{3/2}(1+2^{3/2})~\frac{D_0(\lambda)}{\lambda^{3/2}}\cdot
 e^{-\frac{\lambda}{2}(\ve{r}+\ve{r}\mprime)^2}\Bigg)
\end{eqnarray}

\end{description}

We do now interpret these potentials as vertices of interacting dimer fields -- the physical nature of the fields is
inessential for the following; quite generally, we applied a transformation on a renormalized contact interaction, and
we are now interested in whether or not this transformation, \ie, the RGM averaging over fragment-internal, ``frozen''
degrees of freedom, preserves the renormalized character of amplitudes of the image theory -- whose regularization is
inherited from the renormalized fermion-fermion interaction.

We commence the analysis of the renormalizability of the transformed dimer-dimer theory under the assumption that
the transformation does not affect the power-counting rules. That means, solutions of a Schr\"odinger equation
with and interaction as given by the non-local potentials shall be well-behaved for $\lambda\to\infty$.

\bibliographystyle{unsrt}
\bibliography{Thebibliography.bib}
\end{document}
