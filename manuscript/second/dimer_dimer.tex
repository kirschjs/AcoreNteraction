\documentclass[onecolumn,preprint,superscriptaddress,nofootinbib,notitlepage,10pt,linenumbers]{revtex4-1}

\usepackage{graphicx} 
\usepackage{dashrule}
\usepackage{dcolumn} 
\usepackage{longtable} 
\usepackage{mathtools}
\usepackage{amssymb} 
\usepackage{bbold}
\usepackage{amsmath} 
\usepackage{amsfonts} 
\usepackage{wasysym}
\usepackage{slashed} 
\usepackage[dvipsnames]{xcolor} 
\usepackage{soul}
\usepackage{rotating}
\usepackage{paracol}
%\usepackage{pdflscape}
%=============================================================================
\def\cM{{\widehat{M}}}
\newcommand{\sube}{_\text{\tiny EX}}
\newcommand{\subd}{_\text{\tiny D}}
\def\cS{{\mathcal S}}
\newcommand*{\mprime}{^{\prime}\mkern-1.2mu}
\newcommand*{\mdprime}{^{\prime\prime}\mkern-1.2mu}
\newcommand*{\mtprime}{^{\prime\prime\prime}\mkern-1.2mu}
\newcommand{\red}[1]{\textcolor{red}{#1}}
\newcommand{\green}[1]{\textcolor{green}{#1}} 
\newcommand{\gray}[1]{\textcolor{gray}{#1}} 
\newcommand{\blue}[1]{\textcolor{blue}{#1}} 
\newcommand{\bs}[1]{\boldsymbol{#1}} 
\newcommand{\nopieft}{\mbox{$\slashed{\pi}$EFT}} 
\newcommand{\eftnopi}{\mbox{EFT($\slashed{\pi}$) }} 
\newcommand{\Lag}{{\cal L}}
\newcommand{\la}{\label} 
\newcommand{\be}{\begin{equation}} 
\newcommand{\ee}{\end{equation}} 
\newcommand{\vcl}[1]{\ensuremath{\bar{\boldsymbol{r}}_\text{\tiny #1}}}
\newcommand{\toinf}{\ensuremath{\stackrel{\text{\scalebox{0.6}{$\Lambda\to\infty$}}}{\longrightarrow}}}
\newcommand{\cl}[1]{\ensuremath{\bar{r}_\text{\tiny #1}}}
\newcommand{\vsp}[1]{\ensuremath{\boldsymbol{r}}_\text{\tiny #1}}
\newcommand{\rvec}{{\bs{r}}} 
\newcommand{\xvec}{{\bs{x}}} 
\newcommand{\sgmvec}{\ensuremath{\boldsymbol{\sigma}}} 
\newcommand{\tauvec}{\ensuremath{\boldsymbol{\tau}}}
\newcommand{\fm}{\ensuremath{\,\text{fm}^{-1}}} 
\newcommand{\half}{\frac{1}{2}}
\newcommand{\eg}{\textit{e.g.}\;}
\newcommand{\ie}{\textit{i.e.}\;}
\newcommand{\wrt}{\textit{wrt.}\;}
\newcommand{\etc}{\textit{etc.}\;}
\newcommand{\ve}[1]{\ensuremath{\boldsymbol{#1}}}
\newcommand{\ddrei}[1]{\delta_{\tiny \lambda}^{(3)}\!\big(#1\big)}
\newcommand{\abb}{\ensuremath{2\!+\!1^{(A-1)}}}
\newcommand{\coup}[3]{\left[\,#1\,\otimes\,#2\,\right]^{#3}}
\newcommand{\threej}[6]{ \begin{pmatrix}
   #1 & #2 & #3 \\
   #4 & #5 & #6 
  \end{pmatrix}}
\newcommand{\lam}[1]{\mbox{\ensuremath{\Lambda=#1\,\text{fm}^{-1}}}}   
\newcommand{\figref}[1]{fig.~\ref{#1}}
\newcommand{\tabref}[1]{table~\ref{#1}}
\newcommand{\cc}{\ensuremath C(\lambda)}
\newcommand{\dd}{\ensuremath D(\lambda)}
\newcommand{\abab}{\ensuremath $(ab)\!\!:\!\!(ab)$}
\newcommand{\abca}{\ensuremath $(ab)\wasytherefore(ca)$}
\newcommand{\aff}{\ensuremath a_{\text{\tiny ff}}}
\newcommand{\add}{\ensuremath a_{\text{\tiny dd}}}

\begin{document}

\section{dimer-dimer scattering}

\subsection{Resonating-group description}

The two-fragment resonating-group equation
\be\label{eq.rgm.eqom}
\int\left\lbrace~\phi^*_A\phi^*_B\left(-\frac{\hbar^2}{2\mu}\ve{\Delta}_R-E+\mathcal{V}_{AB}\right)
\mathcal{A}_{AB}\left[\phi_A\phi_B\chi(\ve{R})\right]\right\rbrace d\ve{r}^\text{\tiny internal}_{A,B}=0
\ee
embodies the assumption of a rapid internal motion relative to the slow relative motion between
clusters $A$ and $B$.
Ockham's ansatz for spatially symmetric fragments employs a one-parameter Gaussian
\be\label{eq.rgm.corewfkt}
\phi_A:=e^{-\alpha\sum_{i=1}^A\left(\ve{r}_i-\ve{R}_A\right)^2}\;\;\;;
\begin{array}{l}
     \ve{r}_i\;\;\text{\scriptsize : single-particle coordinates}  \\
     \ve{R}_A\;\;\text{\scriptsize : core centre of mass}
\end{array}\;\;.
\ee
If the two fragments contain identical particle species, antisymmetrization is required between these, \ie,
the inter-cluster antisymmetrizer comprises elements $\mathfrak{P}$ of the symmetric group of the particle labels
in cluster $A$ and $B$ ($\mathfrak{S}_{AB}$),   
\be\la{eq.asymmetr}
\mathcal{A}_{AB}:=\mathbb{1}+\sum_{\mathfrak{p}\in \mathfrak{S}_{AB}}(-1)^{\mathfrak{p}}\,\mathfrak{P}\;\;.
\ee 
The interaction, in contrast, is effective only if it involves particles in different clusters, which are {\bf not} of
the same species. Here, the focus is on zero-range interactions between two and three particles. The latter are one
way to avoid a collapse of the three-body system which would occur with two-body zero-range potentials, only.
\be\la{eq.int}
\mathcal{V}_{AB}=\cc\,\sum_{i\in A\atop j\in B}\,\ddrei{\ve{r}_i-\ve{r}_j}+
\dd\,\sum_{i,j,k\atop i\in A\Rightarrow j\vee k\in B}\ddrei{\ve{r}_i-\ve{r}_j}\,\ddrei{\ve{r}_i-\ve{r}_k}\;\,.
\ee
The subscripted contacts indicate a specific regularization, namely, for one Cartesian dimension
\be\la{eq.delta}
\delta^{(1)}(x)=\lim_{\lambda\to\infty}~\sqrt{\frac{\lambda}{\pi}}\cdot e^{-\lambda x^2}
\ee
The nucleon-nucleon interaction assumes this form in the leading-order of the \eftnopi, where the two coupling strengths
are renormalized, \eg, to a dimer and a trimer binding energy, respectively.
Thereby, the parameter $\alpha$ in \eqref{eq.rgm.corewfkt} is correlated with a physical observable and its $\lambda$-regulator
dependence should, even if the fragment's wave function is not subjected directly to a renormalization constraint, vanish
for $\lim_{\lambda\to\infty}$. We assume from here-on, that all considered fragments a within the range of the applicability of the
microscopic theory. In case of \eftnopi, \eg, we shall assume the existence of a functional correlation between
$\alpha_{^2H,^3He,^4He}=f_{^2H,^3H,^4He}(\aff,B(^3H))$. Analytic forms of these functions $f$ remain obscure,
and we rely on reasonable estimates, \eg, for $\alpha_{^2H}$ the dimer $S$-wave function can be derived in closed form
\be
\phi(r)=\frac{e^{-\frac{r}{\aff}}}{\sqrt{2\pi \aff}r}\;\;.
\ee
Foreseeing the importance of the spacial extend of the fragment's state in light of the non-local character of the effective
interaction, demanding that the average radius ensuing from this form equals that of the ansatz \eqref{eq.rgm.corewfkt}, which
yields:
\be
\alpha_{^2H}=\frac{3}{2}\,\,\aff^{-2}
\ee
where the fermion-fermion scattering length $\aff$ is a converging function of the regulator $\lambda$ if one
enforces a $\lambda$ independent fermion-fermion binding energy to renormalize the interaction, \ie, $\cc$.
To refine this choice, we adjust this value once to obtain a known dimer-dimer results which was obtained in a
microscopic four-body calculation, \eg, to obtain for $\lambda\to\infty$ the result in \eqref{eq.ffddrat} and subsequently
scale the $\alpha$'s of other systems with the size of those relative to the dimer.

The parameter representation $\chi(\ve{R})=\int d\ve{R}\mprime\delta^{(3)}(\ve{R}-\ve{R}\mprime)\chi(\ve{R}\mprime)$
allows for a translation of the inter-cluster antisymmetrizer $\mathcal{A}_{AB}$ into a non-local
integro-differential equation which, in general, assumes the form
\be\la{eq.rgm}
(\hat{T}-E)~\chi(\ve{r})+\mathcal{V}^{(1)}(\ve{r})~\chi(\ve{r})+
\int d^{(3)}\ve{r}\mprime~\mathcal{V}^{(2)}(\ve{r},\ve{r}\mprime,E)~\chi(\ve{r}\mprime)=0
\ee
with the radial coordinates denoting the spatial separation between the two fragment's mass centres. If these fragments are
bound states of the spectrum of the single-particle interaction -- here taken as in \eqref{eq.int} -- and if the energy of the relative motion
between these fragments $E$ is small relative to their binding {\bf and} excitation energies, it is in order to attach physical meaning
to the equation of motion which follows after the averaging over internal degrees of freedom in \eqref{eq.rgm.eqom}. That being, an approximation
of the scattering characteristics of the two cluster.

The effective potentials which feed into this equation are then parametrized by the underlying, microscopic interaction strengths,
\eg, for \eftnopi~LO, $\cc$~and $\dd$. Consequently, the extent to which such a single-particle theory is able to describe
many-body features, \eg, four neutrons assuming a
bound di-neutron, 5-body resonances, 6-helium halos, or 16-oxygen, is then assessed by the dependence of the respective many-body
amplitudes on the short-distance regulator scale $\lambda$.

We commence with the four-body system, dimer-dimer scattering therein.
There are two scenarios which embody the transition between Bose and Fermi dynamics, namely a system of two dimers, each one composed of
the same two species of a fermion. We denote this system as \abab~-- a hypothetical dineutron, for example.
Such a system is by all we know scale invariant, which means that its low-energy behaviour is independent of $\lambda$, and does not
require the additional three-body renormalization constraint manifest in $\dd$.
This arrangement has been studied very thoroughly and indeed, the remarkable universality of the ratio between
dimer-dimer and fermion-fermion attraction -- measured by scattering lengths $a>0$ -- has been numerically
discovered~\cite{petrov_dimerov,petrov_dimerov_nm} and confirmed
in numerous approaches~\cite{Naidon_2016,Elhatisari:2016hui,PhysRevA.79.030501,Rupak:2006jj}:
\be\la{eq.ffddrat}
\frac{\add}{\aff}\approx 0.6\;\;.
\ee

If a third species is present in one of the clusters, the dynamics change fundamentally and lead to a collapse of the
three-body state comprised of these three different equal-mass, two-body-contact-interacting particles.
Universal scale invariance reduces to a discrete scale invariance. Expressed less abstractly, parts of the three-body spectrum
show cyclic behaviour instead of convergence.
We denote this composition as \abca. The dimer-dimer potential of such a four-body system inherits the characteristic three-body
scale through its dependence on $\dd$. This four-body amplitude is then correlated to one two-body and one three-body datum,
while \abab, in the absence of such a three-body scale in the zero-range two-body limit,
depends on a two-body observable, only.

With the above formulas, it is arguably painful, but straight forward to arrive at the specific forms for the effective potentials
as given below. 

\begin{description}
	\item[\abab]
\be\la{eq.local.abab}
\mathcal{V}^{(1)}_{\text{\tiny\abab}}(\ve{r})=
2\,\cc\left(\frac{2\alpha}{2\alpha+\lambda}\right)^{\frac{3}{2}}\cdot
 e^{-\frac{2\alpha\lambda}{2\alpha+\lambda}\ve{r}^2}\;\;,
\ee
\be\la{eq.nonlocal.abab}
\mathcal{V}^{(2)}_{\text{\tiny\abab}}(\ve{r},\ve{r}\mprime,E)=
~8~\alpha^{\frac{3}{2}}\cdot e^{-\alpha\ve{r}\mprime^2}\cdot
\left[\frac{\hbar^2}{2\mu}~(4\alpha^2\ve{r}^2-2\alpha)
\cdot e^{-\alpha\ve{r}^2}+
E\cdot e^{-\alpha\ve{r}^2}
-2~\cc\left(\frac{2\alpha}{2\alpha+\lambda}\right)^{\frac{3}{2}}\cdot
 e^{-\alpha\cdot\frac{2\alpha+3\lambda}{2\alpha+\lambda}\ve{r}^2}\right]\;\;.
\ee

\item[\abca]
\begin{eqnarray}\la{eq.local.abca}
\mathcal{V}^{(1)}_{\text{\tiny\abca}}(\ve{r})&=&
3\cdot \cc\left(\frac{2\alpha}{2\alpha+\lambda}\right)^{\frac{3}{2}}\cdot
 e^{-\frac{2\alpha\lambda}{2\alpha+\lambda}\ve{r}^2}\\
&&+\dd\left(
\left(\frac{2\alpha}{2\alpha+\lambda}\right)^{3}\cdot e^{-\frac{4\alpha\lambda}{2\alpha+\lambda}\ve{r}^2}+
\left(\frac{2\alpha}{\sqrt{(2\alpha+\lambda)^2+2\alpha\lambda}}\right)^{3}\cdot
 e^{-\frac{4\alpha\lambda(\alpha+\lambda)}{4\alpha^2+6\alpha\lambda+\lambda^2}\ve{r}^2}
\right)
\end{eqnarray}


\begin{eqnarray}
\mathcal{V}^{(2)}_{\text{\tiny\abca}}(\ve{r},\ve{r}\mprime,E)=&~8~\alpha^{\frac{3}{2}}\cdot\Bigg(&e^{-\alpha\ve{r}\mprime^2}\cdot
\left[\frac{\hbar^2}{2\mu}~(4\alpha^2\ve{r}^2-2\alpha)
\cdot e^{-\alpha\ve{r}^2}+
E\cdot e^{-\alpha\ve{r}^2}\right]\\
&&-\cc\cdot
 e^{-(\alpha+\lambda)(\ve{r}^2+\ve{r}\mprime^2)-2\lambda\ve{r}\mprime\cdot\ve{r}}
-2~\cc\left(\frac{2\alpha}{2\alpha+\lambda}\right)^{\frac{3}{2}}\cdot
 e^{-\alpha\cdot\left(\ve{r}\mprime^2+\frac{2\alpha+3\lambda}{2\alpha+\lambda}\ve{r}^2\right)}\\
 &&-~\dd\left(\frac{\alpha}{\alpha+\lambda}\right)^{\frac{3}{2}}\cdot
 e^{-\frac{2\alpha^2+4\alpha\lambda+\lambda^2}{2(\alpha+\lambda)}(\ve{r}^2+\ve{r}\mprime^2)-\frac{\lambda^2}{\alpha+\lambda}\ve{r}\cdot\ve{r}\mprime}\\
 &&-\dd\left(\frac{2\alpha(\alpha+\lambda)}{2\alpha^2+3\alpha\lambda+\lambda^2}\right)^{\frac{3}{2}}\cdot
 e^{-\frac{2\alpha^2+5\alpha\lambda+\lambda^2}{2(\alpha+\lambda)}\ve{r}^2-(\alpha+\lambda)\ve{r}\mprime^2-2\lambda\ve{r}\cdot\ve{r}\mprime}\Bigg)
 \la{eq.nonlocal.abca}
\end{eqnarray}

\end{description}

\newpage
\subsection{Solution of the resonating-group equation}
Above, any low-energy two-fragment observable was found encoded in the solution of \eqref{eq.rgm}, which assumes the generic form
\begin{gather}\label{eq.rgm.sglnonloc}
\sum_{n=1}^{N_{\text{loc}}}\hat{\eta}_n~e^{-w_n\ve{R}^2}\chi(\ve{R})-
\sum_{n=1}^{N_{\text{n-loc}}}\int\left\lbrace\hat{\zeta}_n\,e^{-a_n\ve{R}^2-b_n\ve{R}\cdot\ve{R}'-c_n\ve{R}'^2}\right\rbrace\chi(\ve{R}') d\ve{R}'=0\\
\intertext{with$\;\;\;\hat{\eta}_n,\hat{\zeta}_n,w_n,a_n,b_n,c_n\;\;\;\text{dependent upon}\;\;\;\cc,\dd,\alpha,\lambda,E,A,B$~.\nonumber} 
\end{gather}
We identify $n=1$ with the quantities which result in the course of the antisymmetrization and integration from the kinetic part
of \eqref{eq.rgm.eqom}, and therefore have
\be
\hat{\eta}_1=\left(-\frac{\hbar^2}{2\mu}\ve{\Delta}_R-E\right)\;\;,\;\;w_1=0\;\;,\;\;
\hat{\zeta}_1=\left(-\frac{\hbar^2}{2\mu}\ve{\Delta}_R-E\right)\zeta_1\;\;.
\ee
Acting with the derivative on the integral kernel produces the structures as listed 
for the \abab~and \abca~ systems in equations \eqref{eq.local.abab} to \eqref{eq.nonlocal.abca}.
c
\paragraph{In partial waves}

\begin{gather}
\text{Expanding}~\;\chi(\ve{R})=R^{-1}\sum_{lm}\phi_{lm}(R)Y_{lm}(\hat{\ve{R}})~
\text{~and projecting from the left with}\;\;
R\,\int d^2\hat{\ve{R}}~Y^*_{lm}(\hat{\ve{R}})\\
\intertext{before substituting}
e^{-b\ve{R}\cdot\ve{R}'}=4\pi\sum_{LM}i^Lj_L(ib RR')Y^*_{LM}(\hat{\ve{R}})Y_{LM}(\hat{\ve{R}}')\;\;,\;\;
\ve{R}\cdot\ve{R}'=-\sqrt{3}\coup{\ve{R}_p}{\ve{R}\mprime_{-p}}{00}~\text{and~}\ve{r}_m=\sqrt{\frac{4\pi}{3}}rY_{1,m}(\hat{\ve{r}})
\end{gather}
yields for
\begin{description}
	\item[$l\geq 0$]
\begin{subequations}\la{eq.rgm.sglnonloc.pw}
\begin{gather}
0=\left(\frac{\hbar^2}{2\mu}\left[-\partial^2_R+\frac{l(l+1)}{R^2}\right]-E\right)
\phi_{lm}(R)+\sum_{n=2}^{N_{\text{loc}}}\eta_n~e^{-w_nR^2}\phi_{lm}(R)\\
-\int
(4\pi RR\mprime )~\zeta_1 \cdot e^{-a_1R^2-c_1R'^2}
\cdot\Bigg\lbrace
\left[-(4a_1^2R^2+b_1^2R\mprime^2-2a_1)+\frac{l(l+1)}{R^2}-\frac{2\mu}{\hbar^2}E\right]
 i^l j_{l}(ib_1 RR')\nonumber\\
+(4a_1b_1)\cdot RR\mprime\cdot
 \left(i^{\vert l-1\vert} j_{\vert l-1\vert}(ib_1 RR') \vert 2l-1\vert
 \threej{1}{\vert l-1\vert}{l}{0}{0}{0}^2+
 i^{l+1} j_{l+1}(ib_1 RR') (2l+3)
 \threej{1}{l+1}{l}{0}{0}{0}^2\right)\Bigg\rbrace
~\phi_{lm}(R\mprime)~dR\mprime\la{eq.rgm.sglnonloc.pw.tex}\\
-\sum_{n=\red{2}}^{N_{\text{n-loc}}}\zeta_n\int
(4\pi RR\mprime)\,i^l j_l(ib_n RR')\cdot 
e^{-a_nR^2-c_nR'^2}~\phi_{lm}(R\mprime)\,dR\mprime\la{eq.rgm.sglnonloc.pw.noloc}
\end{gather}
\end{subequations}

	\item[$l= 0$ and $b_1=0$]
\begin{subequations}\la{eq.rgm.sglnonloc.sw}
\begin{gather}
0=\left(-\frac{\hbar^2}{2\mu}\partial^2_R-E\right)
\phi_{lm}(R)+\sum_{n=2}^{N_{\text{loc}}}\eta_n~e^{-w_nR^2}\phi_{lm}(R)\\
+\int (4\pi RR\mprime )~\left\lbrace\zeta_1 \cdot e^{-a_1R^2-c_1R'^2}
\left[(4a_1^2R^2-2a_1)+\frac{2\mu}{\hbar^2}E\right]
-\sum_{n=2}^{N_{\text{n-loc}}}\zeta_n\,j_0(ib_n RR')\cdot 
e^{-a_nR^2-c_nR'^2}\right\rbrace~\phi_{lm}(R\mprime)\,dR\mprime\la{eq.rgm.sglnonloc.sw.noloc}
\end{gather}
\end{subequations}
\end{description}
For the two dimer-dimer realizations, we list the coefficients of this equation explicitly in table~\ref{tab.dd.pot.coffs}.

\begin{table}
\setlength{\tabcolsep}{4pt}
\renewcommand{\arraystretch}{1.6}
\caption{\label{tab.dd.pot.coffs}{Defining coefficients for the local and non-local effective RGM interaction to be used in
\eqref{eq.rgm.sglnonloc.pw}~to~\eqref{eq.rgm.sglnonloc.sw.noloc} for the scale-invariant \abab, and the discrete-scale-invariant \abca~
dimer-dimer configurations.
Gaussian widths $w,a,b,c$ have units $\text{length}^{-2}$, the local coupling strengths $\eta$ and $\zeta_1$ scale as energies, and the
non-local $\zeta_{n\geq2}$ as an energy density, energy$\cdot\text{length}^{-3}$.}}
\small\centering
\begin{tabular}{l|cccc}
\multicolumn{5}{c}{\abab}\\
$n$ & $\eta$  & $w$ & &\\
2   & $2\,\cc\left(\frac{2\alpha}{2\alpha+\lambda}\right)^{3/2}$ & $\frac{2\alpha\lambda}{2\alpha+\lambda}$ &  &  \\
\hline
           & $\zeta$ & $a$ & $b$ & $c$ \\
1   & $8~\alpha^{3/2}\left(\frac{\hbar^2}{2\mu}\right)$ & $\alpha$ & $0$ & $\alpha$ \\
2   & $-16~\alpha^{3/2}\,\cc\left(\frac{2\alpha}{2\alpha+\lambda}\right)^{3/2}$ & $\alpha\frac{2\alpha+3\lambda}{2\alpha+\lambda}$ & $0$ & $\alpha$ \\
\hline\hline
\multicolumn{5}{c}{\abca}\\
$n$ & $\eta$  & $w$ & &\\
2   & $3\,\cc\left(\frac{2\alpha}{2\alpha+\lambda}\right)^{3/2}$ & $\frac{2\alpha\lambda}{2\alpha+\lambda}$ &  &  \\
3   & $\dd\left(\frac{2\alpha}{2\alpha+\lambda}\right)^3$ & $\frac{4\alpha\lambda}{2\alpha+\lambda}$ &  &  \\
4   & $\dd\left(\frac{2\alpha}{\sqrt{(2\alpha+\lambda)^2+2\alpha\lambda}}\right)^3$ & $\frac{4\alpha\lambda(\alpha+\lambda)}{4\alpha^2+6\alpha\lambda+\lambda^2}$ &  &  \\
\hline
           & $\zeta$ & $a$ & $b$ & $c$ \\
1   & $8~\alpha^{3/2}\left(\frac{\hbar^2}{2\mu}\right)$ & $\alpha$ & $0$ & $\alpha$ \\
2   & $-8~\alpha^{3/2}\cc$ & $\alpha+\lambda$ & $2\lambda$ & $\alpha+\lambda$ \\
3   & $-16~\alpha^{3/2}\,\cc\left(\frac{2\alpha}{2\alpha+\lambda}\right)^{3/2}$ & $\alpha\frac{2\alpha+3\lambda}{2\alpha+\lambda}$ & $0$ & $\alpha$ \\
4   & $-8~\alpha^{3/2}\dd\left(\frac{\alpha}{\alpha+\lambda}\right)^{3/2}$ & $\frac{2\alpha^2+4\alpha\lambda+\lambda^2}{2(\alpha+\lambda)}$ & $\frac{\lambda^2}{\alpha+\lambda}$ & $\frac{2\alpha^2+4\alpha\lambda+\lambda^2}{2(\alpha+\lambda)}$ \\
5   & $-8~\alpha^{3/2}\dd\left(\frac{2\alpha(\alpha+\lambda)}{2\alpha^2+3\alpha\lambda+\lambda^2}\right)^{3/2}$ & $\frac{2\alpha^2+5\alpha\lambda+\lambda^2}{2\alpha+\lambda}$ & $2\lambda$ & $\alpha+\lambda$ \\
\hline\hline
\end{tabular}
\end{table}

\newpage
\subsection{Limiting cases and renormalizability}

It is in order to consider the following limits: 
\begin{description}
\item[zero-range or contact limit]$\lambda\gg\alpha$
\item[local approximation]
$\int d^{(3)}\ve{r}\mprime~\mathcal{V}^{(2)}(\ve{r},\ve{r}\mprime,E)~\chi(\ve{r}\mprime)\stackrel{E\to 0}{\approx}
\chi(\ve{r})\cdot v^{(2)}(\ve{r})\cdot\int d^{(3)}\ve{r}\mprime~v^{(2)}(\ve{r}\mprime)$~.
\end{description}

Assuming an unnaturally large dimer scale emergent from a relatively short-ranged fermion-fermion interaction,
the zero-range approximation is justified and the ensuing dimer-dimer potentials read:

\begin{description}
	\item[(zero-range)~\abab]
\be
\mathcal{V}^{(1)}_{\text{\tiny\abab}}(\ve{r})=
2\,(2\alpha)^{\frac{3}{2}}~\frac{\cc}{\lambda^{\frac{3}{2}}}\cdot
 e^{-2\alpha\ve{r}^2}\;\;,
\ee
\be
\mathcal{V}^{(2)}_{\text{\tiny\abab}}(\ve{r},\ve{r}\mprime,E)=
~8~\alpha^{\frac{3}{2}}\cdot e^{-\alpha\ve{r}\mprime^2}\cdot
\left[\frac{\hbar^2}{2\mu}~(4\alpha^2\ve{r}^2-2\alpha)
\cdot e^{-\alpha\ve{r}^2}+
E\cdot e^{-\alpha\ve{r}^2}
-2\,(2\alpha)^{\frac{3}{2}}~\frac{\cc}{\lambda^{\frac{3}{2}}}\cdot
 e^{-3\alpha\ve{r}^2}\right]\;\;.
\ee

\item[(zero-range)~\abca]
\begin{eqnarray}
\mathcal{V}^{(1)}_{\text{\tiny\abca}}(\ve{r})&=&
3\,(2\alpha)^{\frac{3}{2}}~\frac{\cc}{\lambda^{\frac{3}{2}}}\cdot
 e^{-2\alpha\ve{r}^2}
 +2\,(2\alpha)^{3}~\frac{\dd}{\lambda^{3}}\cdot
 e^{-4\alpha\ve{r}^2}
\end{eqnarray}


\begin{eqnarray}
\mathcal{V}^{(2)}_{\text{\tiny\abca}}(\ve{r},\ve{r}\mprime,E)=&~8~\alpha^{\frac{3}{2}}\cdot\Bigg(&e^{-\alpha\ve{r}\mprime^2}\cdot
\left[\frac{\hbar^2}{2\mu}~(4\alpha^2\ve{r}^2-2\alpha)
\cdot e^{-\alpha\ve{r}^2}+
E\cdot e^{-\alpha\ve{r}^2}\right]\\
&&-\cc\cdot
 e^{-\lambda(\ve{r}+\ve{r}\mprime)^2}
-2~(2\alpha)^{\frac{3}{2}}\,\frac{\cc}{\lambda^{\frac{3}{2}}}\cdot
 e^{-\alpha\ve{r}\mprime^2-3\alpha\ve{r}^2}\\
 &&-\alpha^{\frac{3}{2}}(1+2^{\frac{3}{2}})~\frac{\dd}{\lambda^{\frac{3}{2}}}\cdot
 e^{-\frac{\lambda}{2}(\ve{r}+\ve{r}\mprime)^2}\Bigg)
\end{eqnarray}

\end{description}

We do now interpret these potentials as vertices of interacting dimer fields -- the physical nature of the fields is
inessential for the following; quite generally, we applied a transformation on a renormalized contact interaction, and
we are now interested in whether or not this transformation, \ie, the RGM averaging over fragment-internal, ``frozen''
degrees of freedom, preserves the renormalized character of amplitudes of the image theory -- whose regularization is
inherited from the renormalized fermion-fermion interaction.

We commence the analysis of the renormalizability of the transformed dimer-dimer theory under the assumption that
the transformation does not affect the power-counting rules. That means, solutions of a Schr\"odinger equation
with and interaction as given by the non-local potentials shall be well-behaved for $\lambda\to\infty$.
Renormalizing the fermion-fermion amplitude yields
\be\la{eq.c0running}
\cc\propto\lambda
\ee
and arguably two scenarios for the three-body parameter:
\begin{align}
\text{Three-body spectrum with one single shallow bound state}~\neq f(\lambda):~&\dd\propto e^{\omega\lambda}\\
\text{Three-body spectrum with a tower of Efimov-type states with the shallowest}~\neq f(\lambda):~&\text{countably infinite poles}
\end{align}

Revisit the effective potentials, considering $\lambda\to\infty$:
\begin{description}
	\item[(zero-range)~\abab]
\be
\mathcal{V}^{(1)}_{\text{\tiny\abab}}(\ve{r})=0
\ee
\be
\mathcal{V}^{(2)}_{\text{\tiny\abab}}(\ve{r},\ve{r}\mprime,E)=
~8~\alpha^{\frac{3}{2}}\cdot e^{-\alpha\ve{r}\mprime^2}\cdot
\left[\frac{\hbar^2}{2\mu}~(4\alpha^2\ve{r}^2-2\alpha)
\cdot e^{-\alpha\ve{r}^2}+
E\cdot e^{-\alpha\ve{r}^2}\right]\;\;.
\ee
In words, the sub-threshold dimer-dimer amplitude depends on the microscopic interaction only through the
character of a single dimer as parametrized with $\alpha$. Although, no analytic form of the functional relation
$\alpha=f(\aleph)$ is known, its existence implies a universal low-energy dimer-dimer system, thereby conforming with
the ``Petrov ratio'' $\add/\aff\approx0.6$. 
\item[(zero-range)~\abca]
\begin{eqnarray}
\mathcal{V}^{(1)}_{\text{\tiny\abca}}(\ve{r})&=&
c_1\texttt{P}[\lambda]\cdot
 e^{-4\alpha\ve{r}^2}
\end{eqnarray}


\begin{eqnarray}
\mathcal{V}^{(2)}_{\text{\tiny\abca}}(\ve{r},\ve{r}\mprime,E)=&~8~\alpha^{\frac{3}{2}}\cdot\Bigg(&e^{-\alpha\ve{r}\mprime^2}\cdot
\left[\frac{\hbar^2}{2\mu}~(4\alpha^2\ve{r}^2-2\alpha)
\cdot e^{-\alpha\ve{r}^2}+
E\cdot e^{-\alpha\ve{r}^2}\right]-c_2\texttt{P}[\lambda]\cdot
 e^{-\frac{\lambda}{2}(\ve{r}+\ve{r}\mprime)^2}\Bigg)
\end{eqnarray}
\texttt{ECCE} this structure, which I want to discuss/have an opinion on.
\end{description}


The specific form of the polynomials depends on the implemented three-body renormalization condition, and so do
the values of the non-equal constants $c_{1,2}$. Yet, regardless of the specific shape, the induced $\lambda$ dependence
will translate into dimer-dimer observables which consequently do not have a well defined $\lim\limits_{\lambda\to\infty}$!

\newpage
\section{Summary of physical insight which can be gained accessed with the above}

\begin{description}
\item[universality of fragment-fragment scattering]
\end{description}	

\bibliographystyle{unsrt}
\bibliography{../Thebibliography.bib}
\end{document}
